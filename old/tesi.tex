%!TEX TS-program = pdflatex
%!TEX encoding = UTF-8 Unicode

\documentclass[12pt,a4paper,twoside,english,italian]{book}

% Usare "oneside" invece di "twoside"
% nelle bozze, per risparmiare carta:
% "twoside" produce diverse pagine bianche
% alla fine dei capitoli.

\usepackage[utf8]{inputenc}

   %%%%%%%%%%%%%%%%%%%%%%%%%%%%%%%%%%%%%%%%%%%%%%%%%%%%%%%%%%%%
   % Se nella tesi si inseriscono dei passi in un'altra       %
   % lingua (inglese, per fissare le idee), si puo' istruire  %
   % il TeX di sillabare quella parte di testo con le regole  %
   % inglesi, invece che italiane. A questo scopo basta       %
   % scrivere                                                 %
   %                                                          %
   %    \documentclass[...,english,italian,...]{...}          %
   %                                                          %
   % al posto di \documentclass[...,italian,...],             %
   % dopodiche' la sillabazione sara' italiana fintanto che   %
   % non si incontra il comando \selectlanguage{english}.     %
   % Per tornare all'italiano si scrive                       %
   % \selectlanguage{italian}                                 %
   %%%%%%%%%%%%%%%%%%%%%%%%%%%%%%%%%%%%%%%%%%%%%%%%%%%%%%%%%%%%

\usepackage{uniudtesi}

\usepackage[nottoc]{tocbibind}

\usepackage{indentfirst}

\graphicspath{{./figure/}}
\usepackage{amsmath,amsfonts,amssymb,amsthm}
\usepackage{latexsym}
\usepackage{natbib_ita}

\newcommand{\N}{\mathbb{N}}
\newcommand{\Z}{\mathbb{Z}}
\newcommand{\Q}{\mathbb{Q}}
\newcommand{\R}{\mathbb{R}}
\newcommand{\C}{\mathbb{C}}

\DeclareMathOperator{\traccia}{tr}
\DeclareMathOperator{\sen}{sen}
\DeclareMathOperator{\arcsen}{arcsen}
\DeclareMathOperator*{\maxlim}{max\,lim}
\DeclareMathOperator*{\minlim}{min\,lim}
\DeclareMathOperator*{\deepinf}{\phantom{\makebox[0pt]{p}}inf}

\newcommand{\varsum}[3]{\sum_{#2}^{#3}\!
   {\vphantom{\sum}}_{#1}\;}
\newcommand{\varprod}[3]{\sum_{#2}^{#3}\!
   {\vphantom{\sum}}_{#1}\;}

  %%%%%%%%%%%%%%%%%%%%%%%%%%%%%%%%%%%%%%%%%%%%%%%%%%%%%%%
  %          Numerazione delle formule                  %
  % Se non specificato altrimenti, il LaTeX numera le   %
  % formule come (capitolo.formula) (per esempio (2.5)  %
  % e` la quinta formula del secondo capitolo).         %
  % Con le istruzioni seguenti invece la numerazione    %
  % diventa (capitolo.sezione.formula) (per esempio     %
  % (3.2.6) e` la sesta formula della seconda sezione   %
  % del terzo capitolo):                                %
  %%%%%%%%%%%%%%%%%%%%%%%%%%%%%%%%%%%%%%%%%%%%%%%%%%%%%%%

%\makeatletter
%\@addtoreset{equation}{section}
%\makeatother
%\renewcommand{\theequation}%
%  {\thesection.\arabic{equation}}


              %%%%%%%%%%%%%%%%%%%%%%%%%%
              % Stile degli enunciati  %
              %%%%%%%%%%%%%%%%%%%%%%%%%%

%%%%%%%%%%%%%%%%%%%%%%%%%%%%%%%%%%%%%%%%%%%%%%%%%%%%%%%%%%%
% Con le dichiarazioni seguenti                           %
% teoremi, definizioni, proposizioni, lemmi e corollari   %
% vengono numerati capitolo per capitolo e con un         %
% contatore unico per tutti (per esempio, se subito dopo  %
% il Teorema 2.1 c'e' una definizione, questa sara'       %
% Definizione 2.2)                                        %
%%%%%%%%%%%%%%%%%%%%%%%%%%%%%%%%%%%%%%%%%%%%%%%%%%%%%%%%%%%

\theoremstyle{plain}
\newtheorem{teorema}{Teorema}[chapter]
\newtheorem{proposizione}[teorema]{Proposizione}
\newtheorem{lemma}[teorema]{Lemma}
\newtheorem{corollario}[teorema]{Corollario}

\theoremstyle{definition}
\newtheorem{definizione}[teorema]{Definizione}
\newtheorem{esempio}[teorema]{Esempio}

\theoremstyle{remark}
\newtheorem{osservazione}[teorema]{Osservazione}

  %%%%%%%%%%%%%%%%%%%%%%%%%%%%%%%%%%%%%%%%%%%%%%%%%%%%%%%%
  % I comandi si usano cosi`:                            %
  %                                                      %
  %   \begin{teorema}[di Pitagora]                       %
  %   La somma dei quadrati ecc.                         %
  %   \end{teorema}                                      %
  %                                                      %
  % Le parole "di Pitagora" fra parentesi quadre         %
  % sono facoltative. Non bisogna inserire               %
  % manualmente degli spazi prima e dopo gli enunciati,  %
  % perche' e` automatico!                               %
  %%%%%%%%%%%%%%%%%%%%%%%%%%%%%%%%%%%%%%%%%%%%%%%%%%%%%%%%


  %%%%%%%%%%%%%%%%%%%%%%%%%%%%%%%%%%%%%%%%%%%%%%%%%%%%%%%%%%%%%%
  % Il pacchetto amsthm definisce anche l'ambiente "proof"     %
  % per le dimostrazioni.                                      %
  % Esempio di uso:                                            %
  %                                                            %
  %   \begin{proof}                                            %
  %   Sia $X$ un insieme ecc.                                  %
  %   \end{proof}                                              %
  %                                                            %
  %%%%%%%%%%%%%%%%%%%%%%%%%%%%%%%%%%%%%%%%%%%%%%%%%%%%%%%%%%%%%%

       %%%%%%%%%%%%%%%%%%%%%%%%%%%%%%%%%%%%%%%%%%%%%%%%%%%%%%%
       %                   makeidx                           %
       %                                                     %
       % Pacchetto per la generazione automatica dell'indice %
       % analitico. Per esempio, se vogliamo che la parola   %
       % "analitico" venga indicizzata nella frase           %
       %                                                     %
       %    "un metodo analitico di soluzione"               %
       %                                                     %
       % bisogna scrivere                                    %
       %                                                     %
       %    "un metodo analitico\index{analitico} di         %
       %              soluzione".                            %
       %                                                     %
       % Compilando il file, il LaTeX produrra' un file      %
       % ausiliario che termina con ".idx". Bisogna far      %
       % processare questo file idx dal programma            %
       % ausiliario "bibtex", che produrra' a sua volta un   %
       % altro file ancora. Dare infine un'ultima passata    %
       % col LaTeX. Si puo' tranquillamente lasciare         %
       % la compilazione dell'indice verso la fine della     %
       % stesura del lavoro, quando tutto e' ormai quasi     %
       % definitivo.                                         %
       %                                                     %
       %%%%%%%%%%%%%%%%%%%%%%%%%%%%%%%%%%%%%%%%%%%%%%%%%%%%%%%

%\usepackage{makeidx}
%\makeindex

% Ridefiniamo la riga di testa delle pagine:
\usepackage{fancyhdr}
\pagestyle{fancy}
\renewcommand{\chaptermark}[1]{\markboth{#1}{}}
\renewcommand{\sectionmark}[1]{\markright{\thesection\ #1}}
\fancyhf{}
\fancyhead[LE,RO]{\bfseries\thepage}
\fancyhead[LO]{\bfseries\rightmark}
\fancyhead[RE]{\bfseries\leftmark}
\renewcommand{\headrulewidth}{0.5pt}
\renewcommand{\footrulewidth}{0pt}
\setlength{\headheight}{14.5pt}

               %%%%%%%%%%%%%%%%%%%%%%%%%%%%%%%%%%%%%%
               %  Informazioni generali sulla Tesi  %
               %    da usare nell'intestazione      %
               %%%%%%%%%%%%%%%%%%%%%%%%%%%%%%%%%%%%%%

  \titolo{Similarità di sottografi nelle reti complesse}
  \titoloeng{Subgraph Similarity\\in Real-World Labeled Networks}
  \laureando{Gaspare Ferraro}
  \annoaccademico{2016-2017}
  \facolta{Scienze Matematiche, Fisiche e Naturali}
  \corsodilaureatriennalein{Informatica}
  \relatore[Prof.]{Roberto Grossi}
  \relatoreDue[Prof.]{Andrea Marino}
% \correlatore{Talaltro dei Tali}
% \correlatoreDue{Secondo Correlatore}
%  \dedica{Ai miei genitori\\
%    per non avermi tagliato i viveri} % (facoltativo)

% Per l'ipertesto:
 \usepackage{hyperref} % gia' caricato da uniudtesi
 \hypersetup{
  % pdfpagelayout=SinglePage, % default
  % pdfpagemode=UseOutlines, % default
  bookmarksopen, % default
  bookmarksopenlevel=2, % default;
  pdftitle={Similarita' di sottografi nelle reti complesse},
  pdfauthor={Gaspare Ferraro},
  pdfsubject={Modello di Tesi di Laurea},
  pdfkeywords={tesi laurea LaTeX}
  }

\begin{document}

\renewcommand{\theequation}{\thechapter.\arabic{equation}}%si torna alle formule numerate come da default
\renewcommand{\thesection}{\thechapter.\arabic{section}}%si torna alle sezioni numerate come da default

\frontmatter

\maketitle

% \renewcommand{\theequation}{\arabic{equation}}%consigliato per migliorare i numeri di equazione nell'introduzione
% \renewcommand{\thesection}{\arabic{section}}%consigliato per migliorare i numeri di equazione nell'introduzione

\tableofcontents

% \listoffigures

\mainmatter
 
 %!TEX TS-program = pdflatex
%!TEX root = tesi.tex
%!TEX encoding = UTF-8 Unicode


       %%%%%%%%%%%%%%%%%%%%%%
       %                    %
       %  Introduzione.tex  %
       %                    %
       %%%%%%%%%%%%%%%%%%%%%%

\chapter{Introduzione}

\section{Definizioni di base e notazione}

\begin{definizione}\label{def:grafo}
	Si dice grafo una coppia ordinata $(V, E)$, con $V$ insieme dei vertici e $E \subseteq V \times V$ insieme degli archi.
\end{definizione}
	
	Se due vertici $u, v \in V$ sono congiunti da un arco si dicono estremi dell'arco, in questo caso indichiamo l'arco con la coppia $(u, v) \in E$.
	
	Se $(u,v) \in E \Leftrightarrow (v,u) \in E$ il grafo si dice non orientato. Considereremo solo grafi non orientati se non specificato altrimenti.
	
	Chiamiamo gli elementi dell'insieme $N(v) = \{ u : (v,u) \in E \}$ vicini del vertice $v$. Il numero dei vicini di $v$ è detto grado di $v$.
	
	Una sequenza di nodi $v_{1}, v_{2}, \ldots, v_{k}$ si dice cammino se $(v_{i}, v_{i+1}) \in E$ $\forall i = 1, \ldots k-1$; un cammino si dice semplice se $v_{i} \neq v_{j}$ $\forall i,j$ $1 \leq i \le j \leq k$.

\begin{definizione}\label{def:sottografo}
	Un grafo $G' = (V', E')$ si dice sottografo di $G = (V, E)$ se $V' \subset V$ e $E' \subset E$.
\end{definizione}
	
	Scrivendo $G' \subset G$ indichiamo che $G'$ è sottografo di $G$.
	
\begin{definizione}\label{def:alfabeto}
	Si dice alfabeto un insieme finito di elementi, chiamati simboli o caratteri.
\end{definizione}
	
	Denotato con $\Sigma$ l'alfabeto, chiamiamo $\Sigma^{k}$ l'insieme di tutte le stringhe lunghe $k$ formate da simboli di $\Sigma$.
	
\begin{definizione}\label{def:grafoetichettato}
	Si dice grafo etichettato una terna ordinata $(V,E,L)$ con $(V,E)$ grafo e $L : V \rightarrow \Sigma$ una funzione che associa ad ogni vertice $v$ un carattere (detto etichetta del vertice) dell'alfabeto $\Sigma$.   
\end{definizione}

	Dato un cammino $\pi = v_{1}, v_{2}, \ldots, v_{k}$,
	estendiamo la funzione $L$ e indichiamo con $L(\pi) = L(v_{1}) \cdot L(v_{2}) \cdot \ldots \cdot L(v_{k})$ la stringa ottenuta concatenando\footnote{Con il simbolo $\cdot$ indichiamo la concatenazione tra due stringhe} le etichette dei singoli nodi del cammino.
	

\section{Il problema}

\section{Applicazioni reali}


 %!TEX TS-program = pdflatex
%!TEX root = tesi.tex
%!TEX encoding = UTF-8 Unicode


     %%%%%%%%%%%%%%%%%%%%
     %                  %
     %  capitolo1.tex   %
     %                  %
     %%%%%%%%%%%%%%%%%%%%

\chapter{Metodi utilizzati}

\section{Color-Coding}

Il Color-Coding è metodo inventato 


\appendix

% %!TEX TS-program = pdflatex
%!TEX root = tesi.tex
%!TEX encoding = UTF-8 Unicode

%%%%%%%%%%%%%%%%%%%%%%

\chapter{Code snippets}
  
The code written for this thesis can be found in the personal GitHub page\footnote{\url{https://github.com/GaspareG/SubgraphSimilarity}}.\medskip

\subsection*{Definitions}

Global definitions of some utilities.

\begin{lstlisting}
typedef long long ll;

typedef COLORSET uint32_t  // COLORSET is a bitset of 32 bit
unsigned int N, Q;         // Numer of nodes and length of paths
int color[N];              // Random coloring of nodes
char label[N];             // Labels of nodes
vector<int> G[N];          // Adjacency list for every node in G
map<COLORSET, ll> M[Q][N]; // Color coding table

// Get p-th bit of colorset n
bool getBit(COLORSET n, int p){ return ((n >> p) & 1) == 1; }

// Set p-th bit of colorset n to 1
COLORSET setBit(COLORSET n, int p){ return n |= 1 << p; }

// Reset p-th bit of colorset n to 0
COLORSET clearBit(COLORSET n, int p){ return n &= ~(1 << p); }

// Complementary colorset of n
COLORSET getCompl(COLORSET n){ return ((1 << q) - 1) & (~n); }
\end{lstlisting}

\clearpage
\subsection*{Similarity Indices}

Algorithms \ref{alg:bray-curtis} and \ref{alg:jaccard}

\begin{lstlisting}
double BCW(set<string> W, map<string, ll> freqA, 
                          map<string, ll> freqB) {
	ll num = 0ll;
	ll den = 0ll;
	for (string x : W) {
		ll fax = freqA[x];
		ll fbx = freqB[x];
		num += 2 * min(fax, fbx);
		den += fax + fbx;
	}
	return (double)num / (double)den;
}
\end{lstlisting}

\begin{lstlisting}
double FJW(set<string> W, map<string, ll> freqA,
                          map<string, ll> freqB, ll R) {
	ll num = 0ll;
	for (string x : W) {
		ll fax = freqA[x];
		ll fbx = freqB[x];
		num += min(fax, fbx);
	}
	return (double)num / (double) R;
}
\end{lstlisting}

\clearpage
\subsection*{Color coding}

Implementation of Algorithm \ref{alg:color-coding}

\begin{lstlisting}
void preprocess() {
	#pragma omp parallel for schedule(guided)
	for (unsigned int u = 0; u < N; u++)
		M[1][u][setBit(0, color[u])] = 1ll;
	
	for (unsigned int i = 2; i <= q; i++) {
		#pragma omp parallel for schedule(guided)
		for (unsigned int u = 0; u < N; u++) {
			for (int v : G[u]) {
				for (auto d : M[i - 1][v]) {
					COLORSET s = d.first;
					ll f = d.second;
					if (getBit(s, color[u])) continue;
					ll fp = M[i][u][setBit(s, color[u])];
					M[i][u][setBit(s, color[u])] = f + fp;
				}
			}
		}
	}
}
\end{lstlisting}

\clearpage
\subsection*{Colorful sampling}

Implementation of Algorithms \ref{alg:colorful-sampler} and \ref{alg:random-path-to}

\begin{lstlisting}
set<string> colorfulSampler(vector<int> X, int r) {
	set<string> W;
	set<vector<int>> R;
	vector<ll> freqX;
	for (int x : X) 
		freqX.push_back(M[q][x][getCompl(0)]);
	discrete_distribution<int> distr(freqX.begin(), freqX.end());
	while (R.size() < (size_t)r) {
		int u = X[distr(eng)];
		vector<int> P = randomPathTo(u);
		if (R.find(P) == R.end()) R.insert(P);
	}
	for (auto r : R) {
		reverse(r.begin(), r.end());
		W.insert(L(r));
	}
	return W;
}
\end{lstlisting}

\begin{lstlisting}
vector<int> randomPathTo(int u) {
	vector<int> P;
	P.push_back(u);
	COLORSET D = getCompl(setBit(0l, color[u]));
	for (int i = q - 1; i > 0; i--) {
		vector<ll> freq;
		for (int v : G[u]) 
			freq.push_back(M[i][v][D]);
		discrete_distribution<int> distr(freq.begin(), freq.end());
		#pragma omp critical
		{
			u = G[u][distr(eng)];
		}
		P.push_back(u);
		D = clearBit(D, color[u]);
	}
	reverse(P.begin(), P.end());
	return ret;
}
\end{lstlisting}
\clearpage
\subsection*{Frequency count}

Implementation of Algorithm \ref{alg:f-count}

\begin{lstlisting}
map<string, ll> FCount(set<string> W, multiset<int> X) {
	set<string> WR;
	for (string w : W) {
		reverse(w.begin(), w.end());
		WR.insert(w);
	}
	
	vector<tuple<int, string, COLORSET>> old;
	for (int x : X)
		if (isPrefix(WR, string(&label[x], 1)))
			old.push_back(make_tuple(x, string(&label[x], 1),
                                  setBit(0, color[x])));
	
	for (int i = q - 1; i > 0; i--) {
		vector<tuple<int, string, COLORSET>> current;
		current.clear();
		#pragma omp parallel for schedule(guided)
		for (int j = 0; j < (int)old.size(); j++) {
			auto o = old[j];
			int u = get<0>(o);
			string LP = get<1>(o);
			COLORSET CP = get<2>(o);
			for (int v : G[u]) {
				if (getBit(CP, color[v])) continue;
				COLORSET CPv = setBit(CP, color[v]);
				string LPv = LP + label[v];
				if (!isPrefix(WR, LPv)) continue;
				#pragma omp critical
				{ 
					current.push_back(make_tuple(v, LPv, CPv)); 
				}
			}
		}
		old = current;
	}
	
	map<string, ll> frequency;
	for (auto c : old) {
		string s = get<1>(c);
		reverse(s.begin(), s.end());
		frequency[s]++;
	}
	return frequency;
}
\end{lstlisting}

\clearpage
\subsection*{Frequency sampling}

Implementation of Algorithm \ref{alg:f-samp}

\begin{lstlisting}
map<pair<int, string>, ll> FSamp(vector<int> X, int r) {
	map<pair<int, string>, ll> W;
	set<vector<int>> R;
	vector<ll> freqX;
	freqX.clear();
	for (int x : X) 
		freqX.push_back(M[q][x][getCompl(0)]);
	discrete_distribution<int> distr(freqX.begin(), freqX.end());
	while( R.size() < (size_t)r) {
		int rem = r - R.size();
		#pragma omp parallel for schedule(guided)
		for(int i=0; i<rem; i++) {
			int u;
			#pragma omp critical
			{
				u = X[distr(eng)];
			}
			vector<int> P = randomPathTo(u);
			#pragma omp critical
			{
				R.insert(P);
			}
		}
	}
	for (auto r : R) {
		reverse(r.begin(), r.end());
		W[make_pair(*r.begin(), L(r))]++;
	}
	return W;
}
\end{lstlisting}

\clearpage
\subsection*{F-Count Bray-Curtis}

Implementation of Algorithm \ref{alg:f-count-bc}
\begin{lstlisting}
double FCountBrayCurtis(set<int> A, set<int> B, int tau) {
	multiset<int> mA(A.begin(), A.end());
	multiset<int> mB(B.begin(), B.end());
	multiset<int> X(A.begin(), A.end());
	X.insert(B.begin(), B.end());
	set<string> W = colorfulSampler(X, tau);
	map<string, ll> freqA = FCount(W, mA);
	map<string, ll> freqB = FCount(W, mB);
	return BCW(W, freqA, freqB);
}
\end{lstlisting}

\subsection*{F-Count Frequency Jaccard}

Implementation of Algorithm \ref{alg:f-count-fj}
\begin{lstlisting}
double FCountFrequencyJaccard(set<int> A, set<int> B, int tau) {
	map<string, ll> freqA, freqB;
	set<int> AB(A.begin(), A.end());
	AB.insert(B.begin(), B.end());
	multiset<int> X(AB.begin(), AB.end());
	set<string> W = colorfulSampler(X, tau);
	ll R = 0;
	for(int x : X)
	{
		multiset<int> ma(&x,&x+1);
		map<string, ll> freqAB = processFrequency(W, ma);
		bool inA = A.find(x) != A.end();
		bool inB = B.find(x) != B.end();
		for (auto w : freqAB) {
			string s = w.first;
			ll f = w.second;
			R += f;
			if (inA) freqA[s] += f;
			if (inB) freqB[s] += f;
		}
	}
	return FJW(W, freqA, freqB, R);
}
\end{lstlisting}

\clearpage
\subsection*{F-Sample Bray-Curtis}

Implementation of Algorithm \ref{alg:f-samp-bc}
\begin{lstlisting}
double FSampleBrayCurtis(set<int> A, set<int> B, int tau) {
	multiset<int> X(A.begin(), A.end());
	X.insert(B.begin(), B.end());
	map<pair<int, string>, ll> freqX = FSamp(X, tau);
	map<string, ll> freqA, freqB;
	set<string> W;
	for (auto w : freqX) {
		int u = w.first.first;
		int s = w.first.second;
		ll f = w.second;
		W.insert(s);
		if (A.find(u) != A.end()) freqA[s] += f;
		if (B.find(u) != B.end()) freqB[s] += f;
	}
	return BCW(W, freqA, freqB);
}
\end{lstlisting}

\subsection*{F-Sample Frequency Jaccard}

Implementation of Algorithm \ref{alg:f-samp-fj}
\begin{lstlisting}
double FSampleFrequencyJaccard(set<int> A, 
                               set<int> B, int tau) {
	set<int> AB(A.begin(), A.end());
	AB.insert(B.begin(), B.end());
	vector<int> X(AB.begin(), AB.end());
	map<pair<int, string>, ll>  freqX = FSample(X, tau);
	map<string, ll> freqA, freqB;
	set<string> W;
	ll R = 0;
	for (auto w : freqX) {
		int u = w.first.first;
		int s = w.first.second;
		int f = w.second;
		W.insert(s);
		R += f;
		if (A.find(u) != A.end()) freqA[s] += f;
		if (B.find(u) != B.end()) freqB[s] += f;
	}
	return FJW(W, freqA, freqB, R);
}
\end{lstlisting}


\noindent

\clearpage

\backmatter

\bibliography{tesi} 

%  \input{biblio}

% \printindex % se si fa l'indice analitico.

\end{document}
