%!TEX TS-program = pdflatex
%!TEX root = tesi.tex
%!TEX encoding = UTF-8 Unicode

\chapter{Introduction}

With the spread of the Internet and more importantly of the social networks, efficient data analysis on graphs becomes increasingly important.
Graphs are a powerful data structure that model in a natural way a lot of information.

%%%%%%%%%%%%%%%%%%%%%%%%%%%%%%%%%%%%%%%%%%%%%%%%%%%%%%%%%%%%%%%%%%%%%%%%%%%%%%%%%%%%%%%%%%%%%%
%%%%%%%%%%%%%%%%%%%%%%%%%%%%%%%%%%%%%%%%%%%%%%%%%%%%%%%%%%%%%%%%%%%%%%%%%%%%%%%%%%%%%%%%%%%%%%

\section{Basic definitions}

\begin{definizione}\label{def:graph}
    A graph is a pair of sets $G=(V,E)$, where $V$ is the set of vertices (or nodes) and $E \subset V \times V$ is the set of edges.
\end{definizione}

If two vertices $u, v \in V$ are connected by an edge they are called extreme of the edge, in this case we denote the edge with the pair $(u, v) \in E$

If $(u,v) \in E \Leftrightarrow (v,u) \in E$ the graph is called undirected, where not specified we will only deal with undirected graphs.

A sequence of nodes  $v_{1}, v_{2}, \ldots, v_{k}$ is called path if $(v_{i}, v_{i+1}) \in E$ $\forall i = 1, \ldots k-1$; a path is called simple if $v_{i} \neq v_{j}$ $\forall i,j$ $1 \leq i < j \leq k$. A cycle is a path where $(v_{k}, v_{1}) \in E$.

We denote by $N(u) = \{ v : (u,v) \in E \}$ the set of neighbors of the vertex $u$, the cardinality of this set is called degree of $u$ (\textit{deg} $u$ = $|N(u)|)$. 

With $N^{<k}(u)$ we indicate the set of vertex connected to $u$ by a simple path of length less than $k$ (note that $N(u)$ = $N^{<2}(u)$).


\begin{definizione}\label{def:subgraph}
    A graph $G' = (V', E')$ is called subgraph of $G=(V,E)$ if $V' \subset V$ and $E' \subset E$. A subgraph is called induced if $E' = (V' \times V') \cap E$.
\end{definizione}

We use $G' \subset G$ to indicate that the graph $G'$ is a subgraph of $G$ and $G' < G$ to indicate that the graph $G'$ is a induced subgraph of $G$.

Note that an induced subgraph $G' = (V', E')$ can be uniquely identified by the set of its vertex $V'$.\\

\begin{definizione}\label{def:labeledgraph}
	A labeled graph is a triple $(V,E,L)$ where $(V,E)$ is a graph and $L : V \rightarrow \Sigma$
	is a function that assign for every node $v$ a symbol of the alphabet $\Sigma$. We call $L(u) \in \Sigma$ label of the node $u$.
\end{definizione}

Given a path $\pi = v_{1}, v_{2}, \ldots, v_{k}$ we extend the function $L$ and we indicate with $L(\pi) = L(v_{1}) L(v_{2}) \ldots L(v_{k}) \in \Sigma^{k}$ the string obtained by the concatenation of the labels of the nodes in the path.\\

In this thesis we mainly focus to analyze complex network: special graph with a non-trivial topology like random graph. Complex network occur in graphs modeling real system like social networks or computer networks and are characterized by a specific structural features:

\begin{definizione}\label{def:power-law-graph}
	We define as \textit{power-law degree distribution} a networks where the degree of a node $u$ follow, for some $\gamma$ (usually $2 < \gamma < 3$), the probability:
	\begin{equation}
		P(deg(u) = k) \sim k^{-\gamma}  
	\end{equation}
\end{definizione}

\begin{figure}[h]
	\centering
	\begin{minipage}[t]{.45\textwidth}
		\centering
		\includegraphics[width=5.8cm,height=3.5cm]{figure/figure-1-1} % TODO
		\caption{Degree distribution of a random network}
	\end{minipage}\hfill
	\begin{minipage}[t]{.45\textwidth}
		\centering 
		\includegraphics[width=5.8cm,height=3.5cm]{figure/figure-1-2} % TODO
		\caption{Degree distribution of a scale-free complex network}
	\end{minipage}
\end{figure}

\begin{figure}[h]
	\centering
	\begin{minipage}[t]{.45\textwidth}
		\centering
		\includegraphics[width=4cm,height=4cm]{figure/figure-1-3}
		\caption{Random network with $|N| = 8$ and $|E| = 13$}
	\end{minipage}\hfill
	\begin{minipage}[t]{.45\textwidth}
		\centering
		\includegraphics[width=3.8cm,height=4cm]{figure/figure-1-4}
		\caption{Complex network with $|N| = 10$ and $|E| = 11$}
	\end{minipage}
\end{figure}
%%%%%%%%%%%%%%%%%%%%%%%%%%%%%%%%%%%%%%%%%%%%%%%%%%%%%%%%%%%%%%%%%%%%%%%%%%%%%%%%%%%%%%%%%%%%%%
%%%%%%%%%%%%%%%%%%%%%%%%%%%%%%%%%%%%%%%%%%%%%%%%%%%%%%%%%%%%%%%%%%%%%%%%%%%%%%%%%%%%%%%%%%%%%%
\section{The problem}

\begin{problema}
Given an undirected labeled graph $G=(V,E,L)$ over an alphabet $\Sigma$, an integer $q$
and two set of nodes $V_{1}, V_{2} \subset V$, we want to estimate the similarity between the two induced subgraphs $V_{1}, V_{2} < G$ based on the labels frequency of simple paths with nodes in $V_{1} \cup N^{<q}(V_{1})$ and $V_{2} \cup N^{<q}(V_{2})$.

\end{problema}

Will discuss about a more formal and rigorous definition of subgraphs similarity in chapter 2.\\

In the definition we use $V_{1} \cup N^{<q}(V_{1})$ and $V_{2} \cup N^{<q}(V_{2})$ instead of simply $V_{1}$ and $V_{2}$ because in a complex graph we also want to keep in mind of the interaction between the subgraph and the external graph.\\

The difficulty we must face is that, in a complex network, the labels can exponentially explode for increasing values of q and $|\Sigma|$ to $|\Sigma|^{q} \gg |V|$ and, even worse, the number of simple paths can exponentially explode to $|V|^{q}$. 
For the simple reason that in complex networks the average separation is very low (the famous idea of \textit{six degrees of separation}).\\

In this thesis we exploit the problem using randomized techniques and parallelization, which makes the problem suitable even for big network. 

%%%%%%%%%%%%%%%%%%%%%%%%%%%%%%%%%%%%%%%%%%%%%%%%%%%%%%%%%%%%%%%%%%%%%%%%%%%%%%%%%%%%%%%%%%%%%%
%%%%%%%%%%%%%%%%%%%%%%%%%%%%%%%%%%%%%%%%%%%%%%%%%%%%%%%%%%%%%%%%%%%%%%%%%%%%%%%%%%%%%%%%%%%%%%
\section{Practical applications}

The problem can be applied to a lot of context.
That is why it is very important to choose the right domains for the values of the $V, E, L, \Sigma, q$:
\begin{itemize}
\item $V$ are out object we want to modeling.
\item $E$ represent the set of interactions, two vertices are connected if exists a relation among them.
\item $L$ and $\Sigma$ are the category that partition $V$, $|\Sigma|$ should not be too high or to low, note that if $|\Sigma| = 1$ the labeling is useless as $V$ is not really partitioned.
\item $q$ should be low as $N^{<q}(u)$ could be a large portion of $G$, (e.g. in Facebook for $q \simeq 4$ we have $N^{<q}(u) \simeq G$)\cite{Facebook}.
\end{itemize}

Furthermore, we have to choose $G1$ and $G2$ in a way that similarity between two groups answer use some real question, like compare to each other two ego networks or two connected components.\\

% A pratical example: given a social network of people connected by friendship relation, where every people are labeled with their favorite musical genre, estimate the similary, in terms of musical tastes, of two different geographic regions.
