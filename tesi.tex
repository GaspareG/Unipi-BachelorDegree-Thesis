%!TEX TS-program = pdflatex
%!TEX encoding = UTF-8 Unicode

\documentclass[12pt,a4paper,twoside,english,italian]{book}

\usepackage[utf8]{inputenc}

\usepackage{uniudtesi}

\usepackage[nottoc]{tocbibind}

\usepackage{indentfirst}
\usepackage[ruled,vlined,linesnumbered]{algorithm2e}

\graphicspath{{./figure/}}
\usepackage{amsmath,amsfonts,amssymb,amsthm}
\usepackage{latexsym}
\usepackage{graphicx} % [demo] is just for the example
\usepackage{wrapfig}

% \usepackage{natbib}

\newcommand{\N}{\mathbb{N}}
\newcommand{\Z}{\mathbb{Z}}
\newcommand{\Q}{\mathbb{Q}}
\newcommand{\R}{\mathbb{R}}

\newcommand{\C}{\mathbb{C}}

\DeclareMathOperator{\traccia}{tr}
\DeclareMathOperator{\sen}{sen}
\DeclareMathOperator{\arcsen}{arcsen}
\DeclareMathOperator*{\maxlim}{max\,lim}
\DeclareMathOperator*{\minlim}{min\,lim}
\DeclareMathOperator*{\deepinf}{\phantom{\makebox[0pt]{p}}inf}

\newcommand{\varsum}[3]{\sum_{#2}^{#3}\!
   {\vphantom{\sum}}_{#1}\;}
\newcommand{\varprod}[3]{\sum_{#2}^{#3}\!
   {\vphantom{\sum}}_{#1}\;}

%\makeatletter
%\@addtoreset{equation}{section}
%\makeatother
%\renewcommand{\theequation}%
%  {\thesection.\arabic{equation}}

%%%%%%%%%%%%%%%%%%%%%%%%%%%%%%%%%%%%%%%%%%%%%%%%%%%%%%%%%%%

\theoremstyle{plain}
\newtheorem{teorema}{Teorema}[chapter]
\newtheorem{proposizione}[teorema]{Proposizione}
\newtheorem{lemma}[teorema]{Lemma}
\newtheorem{corollario}[teorema]{Corollario}

\theoremstyle{definition}
%\newtheorem{definizione}[teorema]{Definizione}
\newtheorem{definizione}[teorema]{Definition}
\newtheorem{esempio}[teorema]{Example}
\newtheorem{problema}[teorema]{Problem}

\theoremstyle{remark}
\newtheorem{osservazione}[teorema]{Osservazione}

%\usepackage[backend=biber]{biblatex}
%\addbibresource{tesi.bib}

% \usepackage{biblatex}
%\addbibresource{tesi.bib} % with extension

\newcommand{\fsamp}{\textsc{F-samp}\xspace}
\newcommand{\base}{\textsc{base}\xspace}
\newcommand{\fcount}{\textsc{F-count}\xspace}


\usepackage{makeidx}
\makeindex

% Ridefiniamo la riga di testa delle pagine:
\usepackage{fancyhdr}
\pagestyle{fancy}
\renewcommand{\chaptermark}[1]{\markboth{#1}{}}
\renewcommand{\sectionmark}[1]{\markright{\thesection\ #1}}
\fancyhf{}
\fancyhead[LE,RO]{\bfseries\thepage}
\fancyhead[LO]{\bfseries\rightmark}
\fancyhead[RE]{\bfseries\leftmark}
\renewcommand{\headrulewidth}{0.5pt}
\renewcommand{\footrulewidth}{0pt}
\setlength{\headheight}{14.5pt}

  \titolo{Similarità di sottografi nelle reti complesse}
  \titoloeng{Subgraph Similarity\\in Complex Networks}
  \laureando{Gaspare Ferraro}
  \annoaccademico{2016-2017}
  %\facolta{}
  %\facolta{Scienze Matematiche, Fisiche e Naturali}
  \corsodilaureatriennalein{Informatica}
  \relatore[Prof.]{Roberto Grossi}
  \relatoreDue[Prof.]{Andrea Marino}
  % \dedica{Ai miei genitori\\per non avermi tagliato i viveri}

% Per l'ipertesto:
 \usepackage{hyperref} % gia' caricato da uniudtesi
 \hypersetup{
  bookmarksopen, % default
  bookmarksopenlevel=2, % default;
  pdftitle={Subgraph Similarity in Complex Networks},
  pdfauthor={Gaspare Ferraro},
  pdfsubject={Tesis of Bachelor Degree in Computer Science},
  pdfkeywords={Tesis Bachelor Degree Computer Science Gaspare Ferraro}
  }
  
\usepackage{listings}
\usepackage{xcolor}
\lstset{language=C++,
                basicstyle=\ttfamily,
                keywordstyle=\color{blue}\ttfamily,
                stringstyle=\color{red}\ttfamily,
                commentstyle=\color{green}\ttfamily,
                morecomment=[l][\color{magenta}]{\#}
}

% \usepackage[none]{hyphenat}

\usepackage{bold-extra}
\usepackage{lmodern}


\begin{document}

\renewcommand{\theequation}{\thechapter.\arabic{equation}}
\renewcommand{\thesection}{\thechapter.\arabic{section}}

\frontmatter

\maketitle

\cleardoublepage

\tableofcontents

% \listoffigures

\mainmatter

% \let\cleardoublepage\clearpage
 
%!TEX TS-program = pdflatex
%!TEX root = tesi.tex
%!TEX encoding = UTF-8 Unicode

\chapter{Introduction}

With the spread of the Internet and more importantly of the social networks, efficient data analysis on graphs becomes increasingly important.
Graphs are a powerful data structure that model in a natural way a lot of information.

%%%%%%%%%%%%%%%%%%%%%%%%%%%%%%%%%%%%%%%%%%%%%%%%%%%%%%%%%%%%%%%%%%%%%%%%%%%%%%%%%%%%%%%%%%%%%%
%%%%%%%%%%%%%%%%%%%%%%%%%%%%%%%%%%%%%%%%%%%%%%%%%%%%%%%%%%%%%%%%%%%%%%%%%%%%%%%%%%%%%%%%%%%%%%

\section{Basic definitions}

\begin{definizione}\label{def:graph}
    A graph is a pair of sets $G=(V,E)$, where $V$ is the set of vertices (or nodes) and $E \subset V \times V$ is the set of edges.
\end{definizione}

If two vertices $u, v \in V$ are connected by an edge they are called extreme of the edge, in this case we denote the edge with the pair $(u, v) \in E$

If $(u,v) \in E \Leftrightarrow (v,u) \in E$ the graph is called undirected, where not specified we will only deal with undirected graphs.

A sequence of nodes  $v_{1}, v_{2}, \ldots, v_{k}$ is called path if $(v_{i}, v_{i+1}) \in E$ $\forall i = 1, \ldots k-1$; a path is called simple if $v_{i} \neq v_{j}$ $\forall i,j$ $1 \leq i < j \leq k$. A cycle is a path where $(v_{k}, v_{1}) \in E$.

We denote by $N(u) = \{ v : (u,v) \in E \}$ the set of neighbors of the vertex $u$, the cardinality of this set is called degree of $u$ (\textit{deg} $u$ = $|N(u)|)$. 

With $N^{<k}(u)$ we indicate the set of vertex connected to $u$ by a simple path of length less than $k$ (note that $N(u)$ = $N^{<2}(u)$).


\begin{definizione}\label{def:subgraph}
    A graph $G' = (V', E')$ is called subgraph of $G=(V,E)$ if $V' \subset V$ and $E' \subset E$. A subgraph is called induced if $E' = (V' \times V') \cap E$.
\end{definizione}

We use $G' \subset G$ to indicate that the graph $G'$ is a subgraph of $G$ and $G' < G$ to indicate that the graph $G'$ is a induced subgraph of $G$.

Note that an induced subgraph $G' = (V', E')$ can be uniquely identified by the set of its vertex $V'$.\\

\begin{definizione}\label{def:labeledgraph}
	A labeled graph is a triple $(V,E,L)$ where $(V,E)$ is a graph and $L : V \rightarrow \Sigma$
	is a function that assign for every node $v$ a symbol of the alphabet $\Sigma$. We call $L(u) \in \Sigma$ label of the node $u$.
\end{definizione}

Given a path $\pi = v_{1}, v_{2}, \ldots, v_{k}$ we extend the function $L$ and we indicate with $L(\pi) = L(v_{1}) L(v_{2}) \ldots L(v_{k}) \in \Sigma^{k}$ the string obtained by the concatenation of the labels of the nodes in the path.\\

In this thesis we mainly focus to analyze complex network: special graph with a non-trivial topology like random graph. Complex network occur in graphs modeling real system like social networks or computer networks and are characterized by a specific structural features:

\begin{definizione}\label{def:power-law-graph}
	We define as \textit{power-law degree distribution} a networks where the degree of a node $u$ follow, for some $\gamma$ (usually $2 < \gamma < 3$), the probability:
	\begin{equation}
		P(deg(u) = k) \sim k^{-\gamma}  
	\end{equation}
\end{definizione}

\begin{figure}[h]
	\centering
	\begin{minipage}[t]{.45\textwidth}
		\centering
		\includegraphics[width=5.8cm,height=3.5cm]{figure/figure-1-1} % TODO
		\caption{Degree distribution of a random network}
	\end{minipage}\hfill
	\begin{minipage}[t]{.45\textwidth}
		\centering 
		\includegraphics[width=5.8cm,height=3.5cm]{figure/figure-1-2} % TODO
		\caption{Degree distribution of a scale-free complex network}
	\end{minipage}
\end{figure}

\begin{figure}[h]
	\centering
	\begin{minipage}[t]{.45\textwidth}
		\centering
		\includegraphics[width=4cm,height=4cm]{figure/figure-1-3}
		\caption{Random network with $|N| = 8$ and $|E| = 13$}
	\end{minipage}\hfill
	\begin{minipage}[t]{.45\textwidth}
		\centering
		\includegraphics[width=3.8cm,height=4cm]{figure/figure-1-4}
		\caption{Complex network with $|N| = 10$ and $|E| = 11$}
	\end{minipage}
\end{figure}
%%%%%%%%%%%%%%%%%%%%%%%%%%%%%%%%%%%%%%%%%%%%%%%%%%%%%%%%%%%%%%%%%%%%%%%%%%%%%%%%%%%%%%%%%%%%%%
%%%%%%%%%%%%%%%%%%%%%%%%%%%%%%%%%%%%%%%%%%%%%%%%%%%%%%%%%%%%%%%%%%%%%%%%%%%%%%%%%%%%%%%%%%%%%%
\section{The problem}

\begin{problema}
Given an undirected labeled graph $G=(V,E,L)$ over an alphabet $\Sigma$, an integer $q$
and two set of nodes $V_{1}, V_{2} \subset V$, we want to estimate the similarity between the two induced subgraphs $V_{1}, V_{2} < G$ based on the labels frequency of simple paths with nodes in $V_{1} \cup N^{<q}(V_{1})$ and $V_{2} \cup N^{<q}(V_{2})$.

\end{problema}

Will discuss about a more formal and rigorous definition of subgraphs similarity in chapter 2.\\

In the definition we use $V_{1} \cup N^{<q}(V_{1})$ and $V_{2} \cup N^{<q}(V_{2})$ instead of simply $V_{1}$ and $V_{2}$ because in a complex graph we also want to keep in mind of the interaction between the subgraph and the external graph.\\

The difficulty we must face is that, in a complex network, the labels can exponentially explode for increasing values of q and $|\Sigma|$ to $|\Sigma|^{q} \gg |V|$ and, even worse, the number of simple paths can exponentially explode to $|V|^{q}$. 
For the simple reason that in complex networks the average separation is very low (the famous idea of \textit{six degrees of separation}).\\

In this thesis we exploit the problem using randomized techniques and parallelization, which makes the problem suitable even for big network. 

%%%%%%%%%%%%%%%%%%%%%%%%%%%%%%%%%%%%%%%%%%%%%%%%%%%%%%%%%%%%%%%%%%%%%%%%%%%%%%%%%%%%%%%%%%%%%%
%%%%%%%%%%%%%%%%%%%%%%%%%%%%%%%%%%%%%%%%%%%%%%%%%%%%%%%%%%%%%%%%%%%%%%%%%%%%%%%%%%%%%%%%%%%%%%
\section{Practical applications}

The problem can be applied to a lot of context.
That is why it is very important to choose the right domains for the values of the $V, E, L, \Sigma, q$:
\begin{itemize}
\item $V$ are out object we want to modeling.
\item $E$ represent the set of interactions, two vertices are connected if exists a relation among them.
\item $L$ and $\Sigma$ are the category that partition $V$, $|\Sigma|$ should not be too high or to low, note that if $|\Sigma| = 1$ the labeling is useless as $V$ is not really partitioned.
\item $q$ should be low as $N^{<q}(u)$ could be a large portion of $G$, (e.g. in Facebook for $q \simeq 4$ we have $N^{<q}(u) \simeq G$)\cite{Facebook}.
\end{itemize}

Furthermore, we have to choose $G1$ and $G2$ in a way that similarity between two groups answer use some real question, like compare to each other two ego networks or two connected components.\\

% A pratical example: given a social network of people connected by friendship relation, where every people are labeled with their favorite musical genre, estimate the similary, in terms of musical tastes, of two different geographic regions.

 
%!TEX TS-program = pdflatex
%!TEX root = tesi.tex
%!TEX encoding = UTF-8 Unicode


     %%%%%%%%%%%%%%%%%%%%
     %                  %
     %  capitolo1.tex   %
     %                  %
     %%%%%%%%%%%%%%%%%%%%

\chapter{Basic tools}

In this chapter we first give a definition of subgraphs similarity 

% In this chapter we first recall some definition on documents similarity and then introduce two different definitions of subgraphs similarity.

%Following we use square brackets [ ] to distinguish multisets from sets, leaving the curly brackets \{ \} to sets 


\section{Similarity indices}

\begin{definizione}\label{def:jaccard}
    Given two set $A$ and $B$ we define the \textbf{Jaccard index} as the ratio between the size of intersection and of the union between the two sets:
    
    \begin{equation}
    J(A,B) = \frac{|A \cap B|}{|A \cup B|}
    \end{equation}
    
\end{definizione}

\begin{definizione}\label{def:bray}
    Given two set $A$ and $B$ we define the \textbf{Bray-Curtis index} as:
    
    \begin{equation}
    BC(A,B) = \frac{2 \times |A \cap B|}{|A| + |B|}
    \end{equation}
    
\end{definizione}

We can easily extended the two previous definition to multiset:

\begin{definizione}\label{def:wjaccard}
    Given two multiset $A = (a_{1}, \ldots, a_{n}) $ and $B = (b_{1}, \ldots, b_{n})$ we define the \textbf{weighted Frequency Jaccard index} as:
    
    \begin{equation}
    FJ(A,B) = \frac{\sum\limits_{i=1}^n { min(a_{i}, b_{i}) } }{\sum\limits_{i=1}^n { max(a_{i}, b_{i}) }}
    \end{equation}
    
\end{definizione}

\begin{definizione}\label{def:wbray}
    Given two multiset $A = (a_{1}, \ldots, a_{n}) $ and $B = (b_{1}, \ldots, b_{n})$ we define the \textbf{Bray-Curtis index} on multiset as:
    
    \begin{equation}
    BC(A,B) = \frac{ 2 \times \sum\limits_{i=1}^n { min(a_{i}, b_{i}) } }{\sum\limits_{i=1}^n {a_{i} + b_{i}}}
    \end{equation}
    
\end{definizione}


\section{Documents similarity}

...

\section{Graphs similarity}

\section{Subgraphs similarity}

...

\section{Sketches}

\subsection{min-wise permutation}
\subsection{bottom-k sketches}

\section{Color Coding}

 
\chapter{Computation of subgraph similarity}

In this chapter, we present four different theoretical algorithms to compute subgraphs similarity as previously defined: an exhaustive enumeration, two similar randomized approaches using the tools described in the previous chapter, and a naive randomized approach as a baseline for comparison.\medskip
	
In the following algorithms, we will make use of parallel instructions, postponing the specific programming choices and the comparison among the different approaches to the next chapter.
	
\section{Indices computation}

Now we illustrate the algorithms to calculate the Frequency Jaccard and Bray-Curtis indices, as they are independent from the next algorithms we will present.\bigskip

As previously seen, instead of iterating over all the strings in $\Sigma^{q}$ we can restrict to $\mathcal{L} \subseteq \Sigma^{q}$, which is the set of all possible $q$-grams found in the $q$-paths of $G$.\medskip 

An additional improvement can be made as follows. If we want to compute the similarity between two set $A, B \subset V$, it is enough to ranging, instead over $\mathcal{L}$, over $\mathcal{W} = \{ x \in \Sigma^{q} : x \in L(A) \text{ or } x \in L(B) \} \subseteq \Sigma^{q}$, as we can easily see that for any $x \in ( \Sigma^{q} \setminus \mathcal{W} )$ both $f_A[x]$ and $f_B[x]$ are equal to zero.\bigskip

Also, we can observe that in the Frequency Jaccard index we do not have to explicitly calculate $f_{A \cup B}[x]$ and its sketch, as the exact value of $R = \Sigma_{x \in \mathcal{W}} f_{A \cup B}[x]$ can be easily calculated from $f_{A}[x] \text{ and } f_{B}[x]$.\bigskip

So we define the following procedures based on \eqref{bray-sub} and \eqref{jaccard-sub}.\medskip

\begin{algorithm}[h]
	\small
	\DontPrintSemicolon
	\SetKwInOut{Input}{Input}
	\SetKwInOut{Output}{Output}
	\Input{$\mathcal{W} = $ dictionary of $q$-grams\\$f_{A}[x] = $ frequency of each $x \in \mathcal{W}$ in $A$\\$f_{B}[x] = $ frequency of each $x \in \mathcal{W}$ in $B$}
	\Output{$BC(A,B) = $ the similarity between $A$ and $B$ according to Bray-Curtis index}
	\BlankLine
	$num \gets 0$\;
	$den \gets 0$\;
	\ForEach{$x \in \mathcal{W}$}{
		$num \gets num + 2 \times \min( f_{A}[x], f_{B}[x] )$\;
		$dem \gets den + f_{A}[x] + f_{B}[x]$\;
	}
	$BC \gets \frac{num}{den}$\;
	\Return{$BC$}
	\caption{\textsc{Bray-Curtis}}
	\label{alg:bray-curtis}
\end{algorithm}

\begin{algorithm}[h]
	\small
	\DontPrintSemicolon
	\SetKwInOut{Input}{Input}
	\SetKwInOut{Output}{Output}
	\Input{$\mathcal{W} = $ dictionary of $q$-grams\\$f_{A}[x] = $ frequency of each $x \in \mathcal{W}$ in $A$\\ $f_{B}[x] = $ frequency of each $x \in \mathcal{W}$ in $B$\\ $R =$ summation of all frequency}
	\Output{$FJ(A,B) = $ the similarity between $A$ and $B$ according to Frequency Jaccard index}
	\BlankLine
	$num \gets 0$\;
	\ForEach{$x \in \mathcal{W}$}{
		$num \gets num + \min( f_{A}[x], f_{B}[x] )$\;
	}
	$FJ \gets \frac{num}{R}$\;
	\Return{$FJ$}
	\caption{\textsc{Frequency-Jaccard}}
	\label{alg:jaccard}
\end{algorithm}

Algorithm 1 and 2 compute the values of $BC(A,B)$ and $FJ(A,B)$, as previously defined, by ranging over the given dictionary of $q$-grams $\mathcal{W}$.

\begin{lemma}
	The execution of \textsc{Bray-Curtis} or \textsc{Frequency-Jaccard} requires $O(|W|)$ time and $O(1)$ space. 	
\end{lemma}

In the next algorithms, we will only focus to compute the values of $\mathcal{W}$, $f_{A}$, $f_{B}$ and $R$.

\clearpage 

\section{Naive approach}

The naive approach consists in enumerating all the possible $q$-grams of simple $q$-paths leading to $u \in A \cup B$. This can be done by starting an exhaustive search for each $u \in A \cup B$.
	
\begin{algorithm}[h]
	\small
	\DontPrintSemicolon
	\SetKwInOut{Input}{Input}
	\SetKwInOut{Output}{Output}
	\Input{$q = $ length of the paths\\$A, B = $ set of nodes to compare}
	\Output{$\mathcal{W} = $ dictionary of $q$-grams\\$f_{A}[x] = $ frequency of each $x \in \mathcal{W}$ in $A$\\ $f_{B}[x] = $ frequency of each $x \in \mathcal{W}$ in $B$\\ $R =$ summation of all frequency}
	\BlankLine
	$R \gets 0$\;
	$\mathcal{W} \gets \emptyset$\;
	$f_{A \cup B} \gets \emptyset$ \quad \;    
	$f_{A} \gets \emptyset$\; 
	$f_{B} \gets \emptyset$\; 
	\BlankLine
	\textbf{parallel} \ForEach{$u \in A \cup B$}{
		$\langle \mathcal{W}_{u}, f_{u} \rangle \gets \textsc{ExhaustiveSearch}(\langle u \rangle, q)$\;
		$\mathcal{W} \gets \mathcal{W} \cup \mathcal{W}_{u}$\;
		$f_{A \cup B} \gets f_{A \cup B} \cup f_{u}$\;
	}
	\BlankLine    
	\ForEach{$\langle u, x \rangle \in f_{A \cup B}$}{ 
		$R \gets R + f_{A \cup B}[\langle u, x \rangle]$\;
		\If{$u \in A$}{
			$f_{A}[x] \gets f_{A}[x] + f_{A \cup B}[\langle u, x \rangle]$\;
		}
		\If{$u \in B$}{
			$f_{B}[x] \gets f_{B}[x] + f_{A \cup B}[\langle u, x \rangle]$\;
		}
	}
	\BlankLine
	\Return{$\langle \mathcal{W}$, $f_{A}$, $f_{B}$, $R \rangle$}
	\caption{\textsc{brute-force}}
	\label{alg:brute-force}
\end{algorithm}

Here we define $\mathcal{W}_{u}$ and $f_{u}$ as, respectively, the dictionary and the frequency of $q$-grams of the $q$-paths leading to the node $u \in A \cup B$.
	 
Thus we compute $\mathcal{W}$ with the property $\mathcal{W} = \Sigma_{u \in A \cup B}{\ \mathcal{W}_{u} }$ and, in the same way, $f_{A \cup B}$ with the property $f_{A \cup B} = \Sigma_{u \in A \cup B}{\ f_{u} }$.
	
The value of $R$ is computed as described at the beginning of the chapter, namely, $R = \Sigma_{x \in \mathcal{W} }{\ f_{A \cup B}[x] }$.

At last, we can compute the value of $f_{A}$ and $f_{B}$ from $f_{A \cup B}$ by looking at the leading nodes and separate the frequencies, depending if it belongs to $A$, $B$ or both.

Note that, as we have to separate the frequencies between $f_{A}$ and $f_{B}$, the type of $f_{A \cup B}$ is not a mapping $ \Sigma^{q} \rightarrow \mathbb{N}$ but instead is a mapping $V \times \Sigma^{q} \rightarrow \mathbb{N}$, where the element in $V$ is the leading node of the $q$-path associated with the $q$-gram.\medskip
	
The values of $FJ(A,B)$ and $BC(A,B)$ computed using this method are exact, and we will use it only to evaluate the precision of the other approaches, as it requires to examine all the possible $O(|\Sigma|^{q})$ $q$-gram with a complexity of $O(|V|^{q})$.\medskip
	
For completeness, we also illustrate the $\textsc{ExhaustiveSearch}$ algorithm that keeps track of the current $q$-path and its relative $q$-gram.
	
\begin{algorithm}[h]
	\small
	\DontPrintSemicolon
	\SetKwInOut{Input}{Input}
	\SetKwInOut{Output}{Output}
	\Input{$\pi = \langle u_{1}, \ldots, u_{|\pi|} \rangle $ current traversing path of length $\leq q$ \\$q = $ length of the paths }
	\Output{$\mathcal{W} =$ dictionary of $q$-grams of $q$-path having $\pi$ as suffix\\$f_{u}[\langle u_{q}, x \rangle] = $ frequency of each $x \in \mathcal{W}$ leading to $u_{q}$}
	\BlankLine
	$\mathcal{W} \gets \emptyset$\;
	$f_{u} \gets \emptyset$ \quad \;    
	\BlankLine
	\If{$|\pi| = q$}{
		$\mathcal{W} \gets \{ L(\pi) \}$\;
		$f_{u}[\langle u_{q}, L(\pi) \rangle] \gets 1$\;
	}
	\Else{
		\ForEach{$v \in N(u_{1}) \setminus \pi$}{
			$\langle \mathcal{W}_{v}, f_{v} \rangle \gets \textsc{ExhaustiveSearch}(\langle v \rangle \cdot \pi, q)$\;
			$\mathcal{W} \gets \mathcal{W} \cup \mathcal{W}_{v}$\;
			$f_{u} \gets f_{u} \cup f_{v}$\;
		}	
	}
	\BlankLine
	\Return{$\langle \mathcal{W}$, $f_{u} \rangle$}
	\caption{$\textsc{ExhaustiveSearch}$}
	\label{alg:exhaustive-search}
\end{algorithm}

Here the symbol $\cdot$ is the concatenation of the paths. Note that we put the node $v$ before the path $\pi$ as we are interested to find all the $q$-paths leading to the node $u$.\medskip
	
In Algorithm 4, when $\pi$ is a $q$-path, we have the base case of the recursion that simply returns $\mathcal{W} = \{ L(\pi) \}$, which is the dictionary composed only by the label of $\pi$, and the frequency $f_{u}[\langle u_{q}, L(\pi) \rangle] = 1$ as we have only one path.
	
When the path $\pi$ is shorted than $q$, we recursively visit all its neighbors, with the new path obtained by prepending the node $v$ to the current path $\pi$.\medskip
	
\clearpage
	
Finally, using $N(u_{1}) \setminus \pi$ we avoid to traverse again the nodes already in the path $\pi$. In this way we restrict the searching only to the simple $q$-paths.
	
\begin{lemma}
	For any two sets of nodes $A, B \subseteq V$, the running time of \textsc{brute-force} requires $O(|V|^{q})$ time and $O(\mathcal{L}) = O(|\Sigma|^{q})$ space.
\end{lemma}

Now we present a little example to better understand our ideas.
	
\begin{esempio}
	We want to compute the similarity between the two nodes $4$ and $3$ in the following graph.	
\end{esempio}

\begin{wrapfigure}{R}{0\textwidth}
	\includegraphics[width=0.35\textwidth]{figure/figure-3-2}
\end{wrapfigure}
	
$\textsc{ExhaustiveSearch}(4, 3)$ returns \medskip

$\mathcal{W}_{4} = \{ abc, bac, bbc, cbc \}$ \medskip
		
$f_{4}[\langle 4, abc \rangle] = 2$ ($\textsc{3-1-0}$ $\textsc{4-1-0}$ $\textsc{3-2-0}$ $\textsc{4-2-0}$)\medskip
		
$f_{4}[\langle 4, bac \rangle] = 2$ ($\textsc{1-4-0}$ $\textsc{2-4-0}$)\medskip
		
$f_{4}[\langle 4, bbc \rangle] = 2$ ($\textsc{3-4-0}$)\medskip
		
$f_{4}[\langle 4, cbc \rangle] = 2$ ($\textsc{3-4-0}$)\bigskip
		
$\textsc{ExhaustiveSearch}(3, 3)$ returns\medskip
		
$\mathcal{W}_{3} = \{ abc, acc, bcc, cbc \}$\medskip
		
$f_{3}[\langle 3, abc \rangle] = 2$ ($\textsc{0-1-3}$, $\textsc{0-2-3}$)\medskip
		
$f_{3}[\langle 3, acc \rangle] = 1$ ($\textsc{0-4-3}$)\medskip
		
$f_{3}[\langle 3, bcc \rangle] = 2$ ($\textsc{1-4-3}$, $\textsc{2-4-3}$)\medskip
		
$f_{3}[\langle 3, cbc \rangle] = 2$ ($\textsc{4-1-3}$, $\textsc{4-2-3}$)\bigskip
		
So the dictionary $\mathcal{W}$ is \medskip
		
$\mathcal{W} = \mathcal{W}_3 \cup \mathcal{W}_4 = \{ abc, acc, bac, bbc, bcc, cbc \}$\bigskip
		
The similarity according to the two indices is \medskip

$BC(\{3\}, \{4\}) = \frac{2 \times ( 2 + 0 + 0 + 0 + 0 + 2 ) }{ 4 + 1 + 2 + 2 + 2 + 4 } = \frac{8}{15}$ \medskip
		
$FJ(\{3\}, \{4\}) = \frac{ 2 + 0 + 0 + 0 + 0 + 2 }{ 4 + 1 + 2 + 2 + 2 + 4 } = \frac{4}{15}$ \medskip
		
\clearpage

\section{Efficient computation}

The main hurdle of our problem is to compute the frequency mapping $f_{X}[\ ]$ for some sets $X \in V$, as it can grow up to a size of $|\Sigma|^{q}$, and its definition requires to explore potentially $|V|^{q}$ $q$-paths.\medskip

We present a random unbiased estimator based on color coding and sketching with the property that it can be computed efficiently even on large networks, and its expected value is the actual similarity index~\cite{SubSim}.\medskip

First, using the color coding we reduce the number of potentially explored $q$-paths from $|V|^{q}$ to $2^{O(q)}|V|$, thus making it feasible for large values of $|V|$, assuming $q = O(\log |V|)$.

Second, instead of calculating the correct value of $f_{X}$, we compute its sketch with a size small compared to $|\Sigma|^{q}$, which is a significant benefit when $|\Sigma|$ or $q$ are large.

\subsection*{$preprocess(G,q)$: Color coding of the $q$-paths}

Now we illustrate how to preprocess the input graph $G=(V,E)$ given an integer $q > 0$, in particular, where $q = O(\log |V|)$.

Note that the preprocessing task is performed independently from the choice of the labeling function $L$ and the subsets $A,B$ to compare. It depends only on the graph $G$ and the value of $q$, so we can execute the preprocessing once and then reuse the same color coding table for different values of $A,B$ or even $L$.\medskip

\begin{algorithm}[h]
		
	\small
	\DontPrintSemicolon
	\SetKwInOut{Input}{Input}
	\SetKwInOut{Output}{Output}
	\Input{$G = (V,E)$ undirected graph with $q$ random colors.}
	\Output{$M = $ dynamic programming table for color coding.}
	\BlankLine
	\textbf{parallel} \lForEach{$u \in V$}{$M_{1,u} = \langle \chi(u), 1 \rangle$}
	\BlankLine
	\For{$i \in \{ 2, 3, \ldots, q\}$}{
		\textbf{parallel} \ForEach{$u \in V$}{
			\ForEach{$v \in N(u)$}{
				\ForEach{$\langle C, f \rangle \in M_{i-1,v}$ such that $\chi(u) \not \in C$}{
					$f' \gets M_{i,u}\left(C \cup \{\chi(u)\}\right)$\;
					$M_{i,u} \gets \langle C \cup \{\chi(u)\}, f' + f \rangle$\;
				}
			}
		}   
	}
	\Return{$M$}	
	\caption{$\textsc{preprocess}$: $\textsc{color-coding}$}
	\label{alg:color-coding}
\end{algorithm}

The next goal is to list all the colorful $q$-paths in $G$ using a dynamic programming approach.\medskip

First of all we assign a random coloring $\chi : V \rightarrow [q]$, so that each node $u \in V$ has a color $\chi(u)$ independently and uniformly chosen from $[q]$. Algorithm 5 build and return a table $M$ of size $q \times |V|$ where $M_{i,j}$ stores the collection of pairs $\langle C, f \rangle$ where  $C \subseteq [q]$ is a color set such that $|C| = i$ and there are $f$ colorful $i$-paths leading to the node $j$.\medskip

Our assumption that $q = O(\log |V|)$ allows us to implement, using bit manipulations, operations on color sets in $O(1)$ time as they fit in a machine words.\medskip

Note that each entry $M_{i, j}$ contains at most $\binom{q}{i}$ sets, each with $i$ colors. Hence the computation of the row $i$ can be done in parallel as it depends only from the row $i-1$ and require $O(|E|\ \binom{q}{i-1})$ time (as we scan all the adjacency list). The entire computation requires thus $O(|E|\ \Sigma_{i=1}^{q}{\binom{q}{i-1}}) = O(|E|\ 2^{q})$ time.\bigskip

For what concern space, the table $M$, as we already said, has a total of $q \times |V|$ entries, each of which contains at most $\binom{q}{i}$ pairs $\langle C, f \rangle$.

Each pair can be stored in $O(1)$ as they are simply $2$ integer, we have a total size of $O(\Sigma_{i=1}^{q}{\Sigma_{j=1}^{|V|}{ \binom{q}{i}}}) = O(|V|\ \Sigma_{i=1}^{q}{\binom{q}{i}}) = O(|V|\ 2^{q})$.\bigskip

\begin{lemma}
	Given an undirected graph $G=(V, E)$ random colored in $[q]$, where $q = O(\log |V|)$, Algorithm 5 (preprocess($G$,$q$)) returns the dynamic programming table $M$ of color coding in $O(|E|\ 2^{q})$ time and $O(|V|\ 2^{q})$ space. 
\end{lemma}

\bigskip

It is not difficult to modify the Algorithm 5 to list also the colorful $q$-grams, printing $L(\pi)$ for each colorful $q$-path $\pi$. This makes the algorithms inefficient, indeed we still have to face with the problem that $\mathcal{L} \sim \Sigma^{q}$.\bigskip

So we will pass to the next step.

\subsection*{$query(A,B)$: Sampling and sketching colorful paths}

Now using the color coding table $M$, and given two set of nodes $A, B$, we want to approximate the values of $BC(A,B)$ and $FJ(A,B)$.\bigskip

As already said, we cannnot explore all the colorful $q$-grams, so our idea is to construct a sample of $\mathcal{L}$, without explicitly calculate it, by sampling $r$ $q$-paths from $M$, where $r < |\mathcal{L}|$ is a user-selectable parameter.

We will use as a sample $r$ $q$-paths  without repetition. This can be also seen as a bottom-r sketch of all the $q$-paths.\bigskip 

\clearpage

Our algorithm for $\textsc{query}(A,B)$ consist of three phases as follows:

\begin{itemize}
	\item Compute a suitable sketch $W \subset \mathcal{L}$ such that $\tau = |W|$ is at most $r$, by sampling $r$ colorful $q$-paths using $M$ and setting $W$ as the sets of the $q$-grams of these paths.
	\item Compute $f_{A}[x]$, $f_{B}[x]$ for each $x \in W$.
	\item Approximate $BC(A,B)$ as $BC_{W}(A,B)$ and $FJ(A,B)$ as $FJ_{W}(A,B)$.
\end{itemize}

Where $BC_{W}(A,B)$ and $FJ_{W}(A,B)$ are defined as:

\begin{equation}\label{bc-w}
	BC_{W}(A,B) = \frac{ 2 \times \Sigma_{x \in W} \min(f_{A}[x], f_{B}[x]) }{ \Sigma_{x \in W} f_{A}[x] + f_{B}[x] }
\end{equation}

\begin{equation}\label{fj-w}
	FJ_{W}(A,B) = \frac{ \Sigma_{x \in W} \min(f_{A}[x], f_{B}[x]) }{ \Sigma_{x \in W} f_{A \cup B}[x] }
\end{equation}

\subsection*{Phase 1: Colorful sampler}

Sampling uniformly using a dynamic programming approach is a topic already covered in the literature, e.g. Martin Dyer in ~\cite{Dyer:2003:ACD:780542.780643} or Eric Vigoda in ~\cite{Vigoda2010LectureNO}. In particular, we are interested in sampling $\tau$ $q$-grams from colorful $q$-paths, leading to nodes belonging to $X$, using the color coding table $M$.\bigskip

As we are dealing with weighted sets (where the weights are frequencies), the sample must depend on the frequencies of the $q$-grams ending in $x \in X$, as in the case of consistent weighted sampling, where more frequent $q$-grams need to be sampled more often.
 
As we do not know a priori the frequency of $q$-grams before sampling, we sample $q$-paths uniformly at random so that the probability of getting a $q$-gram is proportional to the number of paths having that $q$-gram, i.e. its frequency. The uniform sampling of paths can be done by looking at the frequencies of colorful $q$-paths in $M$. We know that the number of colorful $q$-paths ending in $v$ is $M_{q,v}([q])$, hence we extract the starting node $x \in X$ of our weighted random $q$-paths with a probability:\bigskip

\begin{equation}
	p_{X}(v) = \frac{ M_{q,v}([q]) }{ \Sigma_{x \in X}{M_{q, x}([q])} }
\end{equation}

\bigskip

Then we generate a random $q$-path by scanning the color coding table $M$ backward from $q-1$ to $1$, choosing nodes with a probability similar to $p_{X}(v)$ (except that during step $i$ we look at row $i$ and in the complementary of the color set of the current $i$-path).\bigskip

\clearpage

We define out sampling algorithm as shown in Algorithm~\ref{alg:colorful-sampler}.

\begin{algorithm}[h]
	\small
	\DontPrintSemicolon
	\SetKwInOut{Input}{Input}
	\SetKwInOut{Output}{Output}
	\Input{$X =$ multiset of nodes from graph $G$\\$M =$ color coding table for $G$\\$r =$ number of colorful paths to sample.}
	\Output{$W = $ random sample composed by $q$-grams of $q$-paths leading to $x \in X$}
	\BlankLine
	$R \gets \{\}$\;
	\BlankLine
	\textbf{parallel} \For{$j\in [r]$}{
		$u\gets$ randomly chosen $v \in X$ with probability~$p_v = \frac{M_{q,v}([q])}{\sum_{z\in X} M_{q,z}([q])}$\label{line:sample}\;
		$\pi\gets \textsc{random-path-to}(u)$\;
		\lIf{$\pi\not\in R$}{$R \gets R \cup \{ \pi \}$}
		\lElse{$j\gets j-1$ \quad //repeat the step}
	}
	\BlankLine
	\Return{$W = \{ L(\pi) : \pi \in R \}$}
	\BlankLine
	\caption{$\textsc{colorful-sampler}$}
	\label{alg:colorful-sampler}
\end{algorithm}

The procedure $\textsc{random-path-to}$ is defined in Algorithm~\ref{alg:random-path-to}.

\begin{algorithm}[h]
	\small
	\DontPrintSemicolon
	\SetKwInOut{Input}{Input}
	\SetKwInOut{Output}{Output}
	\Input{$M =$ color coding table for $G$\\$u = $ leading node of the path }
	\Output{$\pi = $ random colorful path }	
	$P\gets \langle u \rangle$\;
	$D\gets [q] \setminus \{\chi(u)\}$\;
	\For{$i \in \{q-1,\ldots, 1\}$}{
		$u\gets$ randomly chosen $v \in N(u)$ with probability~$p_{v} = \frac{M_{i,v}(D) }{ \sum_{z\in N(u)} M_{i,z}(D)}$\;
		$P\gets u \cdot P$\;
		$D\gets D\setminus \{\chi(u)\}$\;
	}
	\Return{P}   
	\caption{$\textsc{random-path-to}$}
	\label{alg:random-path-to}
\end{algorithm}

Note that the method $\textsc{random-path-to}$ always finds a colorful $q$-path, as at each step we choose nodes only among the ones that will lead to a colorful $q$-path (i.e. the probability $p_{v}$ is $0$ for nodes that don't lead to a colorful $q$-path). This property is guaranteed by the way the color coding table $M$ is generated by Algorithm 5.

\begin{lemma}
	For any multiset of nodes $X$, 
	Algorithm 6 returns a random sample $W \subset |\Sigma^{q}|$ with a complexity of $O(rq)$ both in time and space, 
	where $q = O(\log |V|)$ and $r < |\mathcal{L}| \leq |\Sigma|^{q}$.
\end{lemma}

\clearpage
\subsection*{Phase 2: Frequency count}

Now that we have a sample $W$, of suitable size, composed by $q$-grams, we are interested, for a multiset of nodes $X$, in calculating $f_{X}[x]$ for each $x \in W$. Algorithm~\ref{alg:f-count}, for steps $i = 1, 2, \ldots, q$, proceeds by expanding in $BFS$ order only the $i$-paths ending in a node $u \in X$ and having $i$-grams that are suffixes of $W$ (this operation can be made more space efficient by using tries or by a binary search in a set of strings).\medskip

We maintain a multiset $T$ of these $i$-grams, each represented by a triple $\langle z, x, C \rangle$ to indicate that exists a $i$-path starting from z and leading to a node $u \in X$ whose $i$-gram is $x$ and its colorset is $C$ (note that the same triple triple $\langle z, x, C \rangle$ can appear more times in $T$ as there might exist multiple paths from $z$ to $u$ labeled with the same $i$-gram $x$).\medskip

Also in this case, considering that the computation for the $i$-grams depends only by the $(i-1)$-grams, we can parallelize the operations for the triple with same length $i$.

\begin{algorithm}[h]
	
	\small
	\DontPrintSemicolon
	\SetKwInOut{Input}{Input}
	\SetKwInOut{Output}{Output}
	\Input{$X =$ multiset of nodes from graph $G$\\$W = $ sample of its colorful $q$-grams}
	\Output{$f_X[x] = $ frequency of each $x \in W$}
	\BlankLine
	$T\gets[\,]$ \quad // step~$i=1$\; 
	\BlankLine
	\textbf{parallel} \ForEach{$u \in X$ such that $L(u)$ appears at the end of a $q$-gram in $W$}{
		$T \gets T \cup [\langle u, L(u), \{\chi(u) \} \rangle]$\;
	}
	\BlankLine
	\For{$i \in \{ 2, 3, \ldots, q\}$}{
		$T' \gets [\,]$\;
		\textbf{parallel} \ForEach{$\langle z, x, C \rangle \in T$}{
			\ForEach{$v \in N(z)$ such that $\chi(v) \not \in C$}{
				\If{$L(v) \cdot x$ is a suffix of a $q$-gram in $W$}{
					$T' \gets T' \cup [\langle v, L(v) \cdot x, C \cup \{\chi(v)\} \rangle]$\;
				}
			}
		}   
		$T \gets T'$\;
	}
	\BlankLine
	$f_X \gets (0,\ldots,0)$\;
	\lForEach{$\langle z, x, C \rangle \in T$}{
		$f_X[x] \gets f_X[x]+1$
	}
	\BlankLine
	\Return{$f_X$}
	\BlankLine
	\caption{\textsc{f-count}: exactly counting frequencies of sampled $q$-grams}
	\label{alg:f-count}
\end{algorithm}

\clearpage

It may happen that, in some big instance, Algorithm 8 explore many colorful paths as it expands the paths in $X \cup N^{<q}(X)$ .

To alleviate this issue we present a modified version of the Algorithm 6, called $\textsc{f-samp}$, that estimate the value of $f_{X}[x]$ after having computed the sketch.\medskip

In Algorithms 9, as we already did in \textsc{Brute-Force}, we keep track of the leading nodes of all the $q$-paths, in this way we can use $f_{X}$ to construct $f_A$, $f_B$ and $f_{A \cup B}$. In addition, we estimate, with the lines $8$ and $9$, the value of $f_X$ using the sampled $q$-paths $R$.\medskip

Of course, this speed up the computation time, on the other hand, as we will see in the next chapter, the accuracy will be affected and we will need a greater value of $r$ to have a better estimation of the similarity indices.

\begin{algorithm}[h]
	\small
	\DontPrintSemicolon
	\SetKwInOut{Input}{Input}
	\SetKwInOut{Output}{Output}
	\Input{$X =$ multiset of nodes from graph $G$\\$M =$ color coding table for $G$\\$r =$ number of colorful paths to sample}
	\Output{$W = $ random sample set of colorful $q$-grams $x \in L(X)$ with probability $p_X(x)$\\$f_{X}[\langle u_{q}, x \rangle] = $ frequency of each $x \in \mathcal{W}$ leading to $u_{q}$}
	$R \gets \{\}$\;
		
	\BlankLine
	\textbf{parallel} \For{$j\in [r]$}{
		$u\gets$ randomly chosen $v \in X$ with probability~$p_v = \frac{M_{q,v}([q])}{\sum_{z\in X} M_{q,z}([q])}$\;
		$\pi\gets \textsc{random-path-to}(u)$\;
		\lIf{$\pi\not\in R$}{$R \gets R \cup \{ \pi \}$}
		\lElse{$j\gets j-1$ \quad //repeat the step}
	}
	\BlankLine
	$W \gets \{ L(\pi) : \pi \in R \}$\;
	\BlankLine
	$f_X \gets (0,\ldots,0)$\;
	\lForEach{$\pi = \langle u_{1}, \ldots, u_{q} \rangle \in R$}
	{
		$f_X[\langle u_{q}, L(\pi) \rangle ] \gets f_X[\langle u_{q}, L(\pi) \rangle]+1$
	}
	\BlankLine
	\Return{$\langle W, f_{X} \rangle$}
	\caption{\textsc{f-samp}}
	\label{alg:f-samp}
\end{algorithm}

\begin{lemma}
	For any multiset of nodes $X$, 
	Algorithm 9 ($\textsc{f-samp}(X, M, r)$) return a random sample $W \subset |\Sigma^{q}|$ and the map frequency $f_{X}[x]$
	with a complexity of $O(rq)$ both in time and space, 
	where $q = O(\log |V|)$ and $r < |\mathcal{L}| \leq |\Sigma|^{q}$.
\end{lemma}

\clearpage

\subsection*{Phase 3: Indices estimation}

Now that we have defined all the generic algorithms, we will use them to estimate both the Bray-Curtis index and the Frequency Jaccard Index.\medskip

The sampling algorithms, $\textsc{colorful-sampler}$ and \fsamp, can be used for estimating both the Bray-Curtis index ($X = A \uplus B$) and the Frequency Jaccard Index ($X = A \cup B$).

In this way, in the Bray-Curtis index, we give more weight of being extracted to the $q$-paths leading to $u \in A \cap B$, as in the multisets union ($\uplus$) we sum the frequency of the elements that belong to the intersection.\medskip

We now present the four final algorithms for estimating both Bray-Curtis index and Frequency Jaccard index, using both \fcount and \fsamp.

\subsection*{F-Count}

For estimating the Bray-Curtis index, using the \fcount approach, we first create the sketch $\mathcal{W}$ by sampling $r$ $q$-grams leading to $X = A \uplus B$, using the $\textsc{Colorful-sampler}$ algorithm. Then we calculate the exact values of $f_{A}[x]$ and $f_{B}[x]$, for each $x \in \mathcal{W}$, using the \fcount algorithm.

Finally, we estimate the real value $BC(A,B)$ with $BC_{ \mathcal{W} }(A,B)$, i.e. the Bray-Curtis index restricted to the strings in $\mathcal{W}$ as defined it in \eqref{bc-w}.

\begin{algorithm}[h]
	\small
	\DontPrintSemicolon
	\SetKwInOut{Input}{Input}
	\SetKwInOut{Output}{Output}
	\Input{$A,B =$ sets of nodes from graph $G$\\$M =$ color coding table for $G$\\$r =$ number of colorful paths to sample}
	\Output{$BC_{W}(A,B) = $ estimation of the Bray-Curtis index between $A$ and $B$ according to the \fcount algorithm  }
	\BlankLine
	$\mathcal{W} \gets \textsc{Colorful-sampler}(A \uplus B, M, r)$ \;
	$f_{A} \gets \textsc{F-count}(A, W)$ \;
	$f_{B} \gets \textsc{F-count}(B, W)$ \;
	\BlankLine
	\Return{$\textsc{Bray-Curtis}(\mathcal{W}, f_{A}, f_{B})$}
	\caption{\textsc{f-count-bc}}
	\label{alg:f-count-bc}
\end{algorithm}

For estimating the Frequency Jaccard index, we create the sketch $\mathcal{W}$ with $X = A \cup B$, always using the $\textsc{Colorful-sampler}$ algorithm. 

Then the values of $f_{A}[x]$ and $f_{B}[x]$ are calculated in a different way, as we also want to calculate the value of $R = \Sigma_{x \in \mathcal{W}} f_{A \cup B}[x]$.

Using the property $R = \Sigma_{u \in A \cup B}\ \Sigma_{x \in \mathcal{W}} f_{u}[x] $ and $f_{X}[x] = \Sigma_{u \in X}{ f_{u}[x] }$, we calculate, for each $u \in A \cup B$, the exact value $f_{u}$: frequency of each $q$-gram leading to $u$, which $q$-gram belong to the sampled dictionary $\mathcal{W}$.

We sum all the frequencies $f_u[x]$, for $x \in \mathcal{W}$, to $R$ and finally merge $f_{u}$ to $f_A$, if $u \in A$, and $f_B$, if $u \in B$.

\begin{algorithm}[h]
	\small
	\DontPrintSemicolon
	\SetKwInOut{Input}{Input}
	\SetKwInOut{Output}{Output}
	\Input{$A,B =$ sets of nodes from graph $G$\\$M =$ color coding table for $G$\\$r =$ number of colorful paths to sample}
	\Output{$FJ_{W}(A,B) = $ estimation of the Frequency Jaccard index between $A$ and $B$ according to the \fcount algorithm  }
	\BlankLine
	$\mathcal{W} \gets \textsc{Colorful-sampler}(A \cup B, M, r)$ \;
	$f_A \gets (0,\ldots,0)$\;
	$f_B \gets (0,\ldots,0)$\;
	$R \gets 0$\;
	\BlankLine
	\ForEach{$u \in A \cup B$}{
		$f_{u} \gets \textsc{F-count}([u], \mathcal{W})$\;
		\BlankLine
		\ForEach{$ x \in \mathcal{W}$}{
			$R \gets R + f_{u}[x]$\;
		}
		\BlankLine
		\If{$u \in A$}{
			$f_A \gets f_A \cup f_{u} $\;
		}
		\BlankLine
		\If{$u \in B$}{
			$f_B \gets f_B \cup f_{u} $\;
		}
	}
	\BlankLine
	\Return{$\textsc{Frequency-Jaccard}(\mathcal{W}, f_{A}[x], f_{B}[x], R)$}
	\caption{\textsc{f-count-fj}}
	\label{alg:f-count-fj}
\end{algorithm}

\subsection*{F-Samp}

Estimating the two indices using the \fsamp algorithm is easier than using \fcount, as \fsamp compute the sketch $\mathcal{W}$ and the frequency map $f_X[x]$ at the same time.\medskip

To estimate the Bray-Curtis index we first call \fsamp with $X = A \uplus B$, then we compute the values of $f_A$ and $f_B$ from $f_X$ by looking if the leading nodes of the $q$-paths belongs to $A$, $B$ or both.

To estimate the Frequency Jaccard index we set $X = A \cup B$ and calculate $R$ using the property $R = \Sigma_{x \in \mathcal{W}}{f_X[x]}$.

\clearpage

\begin{algorithm}[h]
	\small
	\DontPrintSemicolon
	\SetKwInOut{Input}{Input}
	\SetKwInOut{Output}{Output}
	\Input{$A,B =$ sets of nodes from graph $G$\\$M =$ color coding table for $G$\\$r =$ number of colorful paths to sample}
	\Output{$BC_{W}(A,B) = $ estimation of the Bray-Curtis index between $A$ and $B$ according to the \fsamp algorithm  }
	$\langle \mathcal{W}, f_X \rangle \gets \textsc{f-samp}(A \uplus B, M, r)$ \;
	$f_A \gets (0,\ldots,0)$\;
	$f_B \gets (0,\ldots,0)$\;
	\ForEach{$\langle u, x \rangle \in f_X$}{
		\lIf{$u \in A$}{$f_A[x] \gets f_A[x] + f_X[\langle u, x \rangle]$}
		\lIf{$u \in B$}{$f_B[x] \gets f_B[x] + f_X[\langle u, x \rangle]$}
	}
	\Return{$\textsc{Bray-Curtis}(\mathcal{W}, f_{A}, f_{B})$}
	\caption{\textsc{f-samp-bc}}
	\label{alg:f-samp-bc}
\end{algorithm}

\begin{algorithm}[h]
	\small
	\DontPrintSemicolon
	\SetKwInOut{Input}{Input}
	\SetKwInOut{Output}{Output}
	\Input{$A,B =$ sets of nodes from graph $G$\\$M =$ color coding table for $G$\\$r =$ number of colorful paths to sample}
	\Output{$FJ_{W}(A,B) = $ estimation of the Frequency Jaccard index between $A$ and $B$ according to the \fsamp algorithm  }
	$\langle \mathcal{W}, f_X \rangle \gets \textsc{f-samp}(A \cup B, M, r)$ \;
	$f_A \gets (0,\ldots,0)$\;
	$f_B \gets (0,\ldots,0)$\;
	$R \gets 0$\;
	\ForEach{$\langle u, x \rangle \in f_X$}{
		$R \gets R + f_X[\langle u, x \rangle]$\;
		\lIf{$u \in A$}{$f_A[x] \gets f_A[x] + f_X[\langle u, x \rangle]$}
		\lIf{$u \in B$}{$f_B[x] \gets f_B[x] + f_X[\langle u, x \rangle]$;}
	}
	\Return{$\textsc{Frequency-Jaccard}(\mathcal{W}, f_{A}, f_{B}, R)$}
	\caption{\textsc{f-samp-fj}}
	\label{alg:f-samp-fj}
\end{algorithm}

\subsection*{Conclusion}
We have shown how to estimate the two Bray-Curtis and Frequency Jaccard similarity indices using the two approaches \fcount and \fsamp, in particular, as demonstrated in \cite{SubSim}, both $BC_{W}(A,B)$ and $FJ_{W}(A,B)$ are, respectively, unbiased estimators for $BC(A,B)$ and $FJ(A,B)$, i.e. the following definitions shall apply: $BC(A,B) = \mathbb{E}[BC_{W}(A,B)]$ and $FJ(A,B) = \mathbb{E}[FJ_{W}(A,B)]$ for every possible choice of $|W| = 1$.

\clearpage

\section{Baseline algorithm}

In order to validate the effectiveness of our approaches, we compare the previously seen algorithms against a naive randomized approach, the baseline algorithm $\textsc{base}$ that find random paths in a simple way.

\begin{algorithm}[h]
		
	\small
	\DontPrintSemicolon
	\SetKwInOut{Input}{Input}
	\SetKwInOut{Output}{Output}
	\Input{$X =$ array of nodes from graph $G$\\$r =$ number of paths to sample}
	\Output{$W =$ dictionary of $q$-grams sampled\\$f_X[x] = $ frequency of each $x \in W$, where $W = $ naive random sample multiset of $q$-grams for $X$.}
	\BlankLine
	$R\gets\{\}$\;
	\BlankLine
	\textbf{parallel} \For{$j\in [r]$}{
		$u\gets$ randomly chosen $v \in X$ with uniform probability\;
		$P\gets \textsc{naive-random-path-to}(u)$\;
		\lIf{$P \neq \mathtt{null}$ and $P\not\in R$}{$R \gets R \cup \{ P \}$}
		\lElse{$j\gets j-1$ \quad //repeat the step}
	}
	\BlankLine
	$W \gets [ L(P) : P \in R ]$\;
	$f_X \gets (0,\ldots,0)$\;
	\BlankLine
	\lForEach{$x \in W$}{
		$f_X[x] \gets f_X[x]+1$
	}
	\BlankLine
	\Return{$\langle W, f_X \rangle$}
	\BlankLine
	\caption{\textsc{base}\xspace, the baseline sampler}
	\label{alg:base}
\end{algorithm}
	
And the algorithm \textsc{naive-random-path-to}:
	
\begin{algorithm}[h]
	\small
	\DontPrintSemicolon
	\SetKwInOut{Input}{Input}
	\SetKwInOut{Output}{Output}
	\Input{$u = $ leading node of the path }
	\Output{$\pi = $ random $q$-path leading to $u$ or $\mathtt{null}$ } 
	$\pi\gets \langle u \rangle$\;
	\For{$i \in \{q-1,\ldots, 1\}$}{
		\lIf{$N(u) \setminus \pi = \emptyset$}{return $\mathtt{null}$}
		$u\gets$ randomly chosen $v \in N(u) \setminus \pi$ with uniform prob.\;
		$\pi\gets u \cdot \pi$\;
	}
	\Return{$\pi$}   
		
	\caption{\textsc{naive-random-path-to}}
	\label{alg:naive-random-path-to}
\end{algorithm}

Note that, because this is a naive approach, the \textsc{naive-random-path-to} may fail to find a $q$-path leading to $u$ as it goes to explore dead-end paths.

Also in this case we estimate $BC(A,C)$ using $X = A \uplus B$ and $FJ(A,B)$ using $X = A \cup B$ and $R = \Sigma_{x \in W} f_{X}[x]$.

\clearpage

\chapter{Project development}

To confirm the validity, both in terms of correctness and performance, of our algorithms we implemented all the procedures previously illustrated. 
The most important parts of the code can be found in the appendix of this thesis.

\section{Implementation choices and steps}

The algorithms have been implemented using the C++ programming language, 
as it provides good performance in practice and a lot of well implemented data structures in the Standard Template Library.\medskip

We have first implemented the $\textsc{brute}$ algorithm as it was the simpler and give us the correct answers, then we have implemented the three algorithms $\textsc{f-cont}$, $\textsc{f-samp}$, $\textsc{base}$ and check if some practical test gives us some reasonable values.
After making sure that all correctly works, we pass to parallelize them.\medskip

The parallelization has been implemented using the OpenMP API~\cite{openmp}, which defines a simple and flexible interface for developing parallel applications, in particular, we use it to manage the parallel for-loops and the critical sections.\medskip

To make the tests repeatable we used random generators with fixed seed, the subgraphs $A$ and $B$ were generated in three different ways: two random and independent subsets of $V$, two connected components (generated by choosing two random nodes and then expanding them through a $\textsc{BFS}$), two ego-networks.\medskip

All the code was written in a modular and highly customizable way in order
to better test the various algorithms, in the results we explicitly show the parameter used to execute the tests.

\clearpage

\section{Tuning the parameters in practice}

The problem can be applied to a lot of context.
That is why it is very important to choose the right domains for the values of the $V, E, L, \Sigma, q$:
\begin{itemize}
	\item $V$ are out object we want to modeling.
	\item $E$ represent the set of interactions, two vertices are connected if exists a relation among them.
	\item $L$ and $\Sigma$ are the category that partition $V$, $|\Sigma|$ should not be too high or too low, note that if $|\Sigma| = 1$ the labeling is useless as $V$ is not really partitioned.
	\item $q$ should be low as $N^{<q}(u)$ could be a large portion of $G$, (e.g. in Facebook for $q \simeq 4$ we have $N^{<q}(u) \simeq G$)~\cite{Facebook}.
\end{itemize}

\section{Dataset}

For the experiments we use two different kinds of dataset, a small one so we can easily brute-force the real indices and compare the relative errors, and a big one in order to benchmark the performance of the different approaches on a real world complex network.

\paragraph*{NetInf} This graph represents the flow of information on the web among blogs and news websites. The graph was computed by the \textit{NetInf} approach, as part of the \textit{SNAP} project~\cite{netinf}, by tracking cascades of information diffusion to reconstruct ``who copies who'' relationships.

\begin{itemize}
	\item $V$ is the set of blog or news website, $|V| = 854$.
	\item $E$, each website is connected to those who frequently copy their content, $|E| = 3824$.
	\item $\Sigma$ is the set of ranking class of websites ($0$ top $4\%$, $1$ next $15\%$, $2$ next $30\%$, $3$ last $51\%$), $|\Sigma| = 4$.
	\item $L$, each website is labeled according to its importance, using Amazon's Alexa ranking~\cite{alexarank}.
\end{itemize}

\textsl{Considered query:} compute the similarity of two websites $a$ and $b$ or two sets of websites.

\paragraph*{IMDb} In this graph, taken from the \textit{Internet Movie Database}~\cite{imdb} we have:

\begin{itemize}
	\item $V$ is the set of all movies in \textit{IMDb},  $|V| = 1\,060\,209$.
	\item $E$, two movies are connected if their casts share at least one actor, $|E| = 288\,008\,472$.
	\item $\Sigma$ is the set of movies genre, $|\Sigma| = 36$.
	\item $L$, each movie is labeled with its principal genre.
\end{itemize}

\textsl{Considered query:} similarity of actors' ego-networks. Given two actors $a$ and $b$, let $A$ and $B$ be their ego-networks, i.e., the sets of nodes corresponding to movies in which respectively $a$ and $b$ starred, compute the similarity of $A$ and $B$.\medskip

The way we generate the $\textsc{IMDb}$ graph is an example of collaboration graph and is known in literature as Co-stardom network. 

Another famous example is the collaboration graph of mathematicians, where two mathematicians are connected if they have co-authored a paper. 
This collaboration graph is also known as Erdős collaboration graph~\cite{BATAGELJ2000173}, in honor of the famous mathematician Paul Erdős, in this graph is defined also the \textit{Erdős number} as the distance in term of collaboration between Paul Erdős and another person.

\section{Experimental results}

We describe the experimental evaluation for our approach. Our computing platform is a machine with Intel(R) Xeon(R) CPU E5-2620 v3 at 2.40GHz, 24 virtual cores, 128 Gb RAM, running Ubuntu Linux v.4.4.0-22-generic. Code written in C++17, compiled with g++ v.5.4.1 and OpenMP 4.5.\medskip

To better analyze the different approaches described, we take several kinds of experiment in each of them~\footnote{Unless otherwise stated, all the results are the average of $100$ identical experiments, in order to reduce the possible errors randomly caused by the machine.}, all times are expressed in milliseconds.\medskip

An important fact of which to take into account is that we make large use of parallelization, 
so all the running time scale (approximately) linearly on the number of CPU cores used.

\subsection*{Running time}

In this experiment we compare the different running time, of all the parts, from all algorithms. Note that this is an important experiments, as in the real application time is crucial factor.\medskip

First of all we test how much we can go up in \textsc{Brute-Force} with the value of $q$ and the sample size, as this is our bottleneck to analyze the relative errors for the approximated methods.\medskip 

\begin{table}[h]
	\centering
	\begin{tabular}{|c|c|c|c|}
		\hline
		\textsc{Dataset} & $q$ & $|A \cup B|$ & \textsc{Brute-Force} \\ \hline \hline
		\textsc{NetInf}  & $4$ & $100$        & $200$                \\ \hline
		\textsc{NetInf}  & $4$ & $200$        & $420$                \\ \hline
		\textsc{NetInf}  & $4$ & $500$        & $870$                \\ \hline \hline
		\textsc{NetInf}  & $5$ & $100$        & $1\,206$             \\ \hline
		\textsc{NetInf}  & $5$ & $200$        & $2\,736$             \\ \hline
		\textsc{NetInf}  & $5$ & $500$        & $6\,080$             \\ \hline \hline
		\textsc{NetInf}  & $6$ & $100$        & $22\,715$            \\ \hline
		\textsc{NetInf}  & $6$ & $200$        & $49\,828$            \\ \hline
		\textsc{NetInf}  & $6$ & $500$        & $104\,129$           \\ \hline
	\end{tabular}
\end{table}

As expected, we can see that the running time for the bruteforce approach is linear in the size of $|A \cup B|$ and exponential in the value of $q$.\medskip

The second bottleneck for our algorithms is the preprocessing time for the dynamic programming table of color-coding, so we test for both the dataset how can we go up with the value of $q$. Always remembering from initial assumptions that the value of $q$ should not be too high. 

\begin{table}[h]
	\centering
	\begin{tabular}{|c|c|c|}
		\hline
		\textsc{Dataset} & $q$  & \textsc{Color-Coding} \\ \hline \hline
		\textsc{NetInf}  & $7$  & $20$                  \\ \hline
		\textsc{NetInf}  & $9$  & $80$                  \\ \hline
		\textsc{NetInf}  & $11$ & $185$                 \\ \hline
		\textsc{NetInf}  & $13$ & $340$                 \\ \hline
		\textsc{NetInf}  & $15$ & $1\,433$              \\ \hline \hline
		\textsc{IMDb}    & $3$  & $48\,220$             \\ \hline
		\textsc{IMDb}    & $4$  & $105\,943$            \\ \hline
		\textsc{IMDb}    & $5$  & $241\,224$            \\ \hline
		\textsc{IMDb}    & $6$  & $557\,481$            \\ \hline
	\end{tabular}
\end{table}

We can observe that, even in \textsc{IMDb} dataset, the value of $q$ could go high as expected, always remaining under $10$ minutes of running time. \medskip

To better understand these values, the official \textsc{IMDb} statistic~\cite{imdbstat} told us that, out of $1\,837\,357$ actors analyzed, $1\,579\,193$ ($\sim86\%$) are distant $q=3$ from the actor \textit{Kevin Bacon} and $1\,795\,352$ ($\sim98\%$) are distant $q=6$.\medskip

Finally, we test the running time for the different approaches for different value of $q$ and number $R$ of $q$-paths sampled. 

\begin{table}[h]
	\centering
	\begin{tabular}{|c|c|c|c|c|c|c|c|}
		\hline
		\textsc{Dataset} & $q$ & $|A|$ & $|B|$ & $R$      & \textsc{F-Count} & \textsc{F-Sample} & \textsc{Base} \\ \hline \hline
		\textsc{NetInf}  & $3$ & $100$ & $100$ & $1\,000$ & $20$             & $4$               & $2$           \\ \hline
		\textsc{NetInf}  & $3$ & $100$ & $100$ & $5\,000$ & $60$             & $30$              & $15$          \\ \hline
		\textsc{NetInf}  & $5$ & $100$ & $100$ & $1\,000$ & $2\,682$         & $426$             & $3$           \\ \hline
		\textsc{NetInf}  & $5$ & $100$ & $100$ & $5\,000$ & $4\,767$         & $784$             & $20$          \\ \hline
		\textsc{NetInf}  & $7$ & $100$ & $100$ & $100$    & $5\,455$         & $4$               & $2$           \\ \hline
		\textsc{NetInf}  & $7$ & $100$ & $100$ & $200$    & $16\,634$        & $197$             & $2$           \\ \hline \hline
		\textsc{IMDb}    & $3$ & $10$  & $10$  & $100$    & $5\,035$         & $66$              & $1$           \\ \hline
		\textsc{IMDb}    & $4$ & $10$  & $10$  & $1000$   & $/$              & $2\,829$          & $14$          \\ \hline
		\textsc{IMDb}    & $5$ & $10$  & $10$  & $1000$   & $/$              & $4\,739$          & $20$          \\ \hline
		\textsc{IMDb}    & $6$ & $10$  & $10$  & $1000$   & $/$              & $9\,783$          & $36$          \\ \hline
	\end{tabular}
\end{table}

The running time of \base is always extremely low, 
unlike \fcount that, as we already anticipated, is not suitable for sampling to many $q$-paths .
Instead the running time \fsamp results affordable for all the instance. 

This is because both the \fsamp and \base have a complexity of $O(rq)$ while 
\fcount, that analyze the $q$-paths inside $A \cup N^{<q}(A)$ and $B \cup N^{<q}(B)$, could possibly traverse all the graph.

\subsection*{Relative error and variance}

In this experiment we will compare, for increasing value of $R$, how accurate and stable are the different algorithms (using the $\textsc{NetInfo}$ dataset).\medskip

The accuracy is calculated with the average of the relative error between the exact solution of $\textsc{brute}$ and the analyzed algorithm \fcount, \fsamp or \base, instead the stability is calculated as the variance of the results $100$ experiments.

\begin{table}[h]
	\centering
	\begin{tabular}{|c|c|c|c|c|c|c|c|}
		\hline
		$q$ & $|A|$ & $|B|$ & $R$      & $\epsilon_{BC}$ & $\textsc{VAR}_{BC}$ & $\epsilon_{FJ}$ & $\textsc{VAR}_{FJ}$ \\ \hline \hline
		$3$ & $100$ & $100$ & $10$     & $0.02617187$    & $0.00082663$        & $0.02431909$    & $0.000190515$       \\ \hline
		$3$ & $100$ & $100$ & $100$    & $0.02258048$    & $0.00003059$        & $0.02324100$    & $0.000007628$       \\ \hline
		$3$ & $100$ & $100$ & $1\,000$ & $0.03952676$    & $0.00000070$        & $0.04030510$    & $0.000000132$       \\ \hline \hline
		$4$ & $100$ & $100$ & $10$     & $0.03828302$    & $0.00127922$        & $0.03703453$    & $0.000341645$       \\ \hline
		$4$ & $100$ & $100$ & $100$    & $0.01232044$    & $0.00005457$        & $0.00939392$    & $0.000016680$       \\ \hline
		$4$ & $100$ & $100$ & $1\,000$ & $0.01810665$    & $0.00000240$        & $0.02072427$    & $0.000000750$       \\ \hline \hline
		$5$ & $100$ & $100$ & $10$     & $0.04120389$    & $0.00199562$        & $0.04766193$    & $0.000590912$       \\ \hline
		$5$ & $100$ & $100$ & $100$    & $0.01418216$    & $0.00021613$        & $0.01550921$    & $0.000045352$       \\ \hline
		$5$ & $100$ & $100$ & $1\,000$ & $0.02144092$    & $0.00000647$        & $0.02015720$    & $0.000018239$       \\ \hline
		
	\end{tabular}
	\caption{Relative error and variance of the \fcount approach}	
\end{table}

\begin{table}[h]
	\centering
	\begin{tabular}{|c|c|c|c|c|c|c|c|}
		\hline
		$q$ & $|A|$ & $|B|$ & $R$      & $\epsilon_{BC}$ & $\textsc{VAR}_{BC}$ & $\epsilon_{FJ}$ & $\textsc{VAR}_{FJ}$ \\ \hline \hline
		$3$ & $100$ & $100$ & $10$     & $0.53290243$    & $0.02463258$        & $0.64549929$    & $0.01098586$        \\ \hline
		$3$ & $100$ & $100$ & $100$    & $0.26679417$    & $0.00291718$        & $0.35897713$    & $0.00141635$        \\ \hline
		$3$ & $100$ & $100$ & $1\,000$ & $0.05437719$    & $0.00023471$        & $0.12111130$    & $0.00015040$        \\ \hline \hline
		$4$ & $100$ & $100$ & $10$     & $0.56332930$    & $0.03922519$        & $0.71466000$    & $0.01504646$        \\ \hline
		$4$ & $100$ & $100$ & $100$    & $0.42694364$    & $0.00315346$        & $0.58182255$    & $0.00148827$        \\ \hline
		$4$ & $100$ & $100$ & $1\,000$ & $0.17956068$    & $0.00028600$        & $0.26846896$    & $0.00016087$        \\ \hline \hline
		$5$ & $100$ & $100$ & $10$     & $0.56603667$    & $0.03097334$        & $0.72217070$    & $0.01117576$        \\ \hline
		$5$ & $100$ & $100$ & $100$    & $0.60814392$    & $0.00324602$        & $0.73568322$    & $0.00098974$        \\ \hline
		$5$ & $100$ & $100$ & $1\,000$ & $0.37832023$    & $0.00030943$        & $0.49424173$    & $0.00010248$        \\ \hline
	\end{tabular}
	\caption{Relative error and variance of the \fsamp approach}	
\end{table}

\begin{table}[h]
	\centering
	\begin{tabular}{|c|c|c|c|c|c|c|c|}
		\hline
		$q$ & $|A|$ & $|B|$ & $R$      & $\epsilon_{BC}$ & $\textsc{VAR}_{BC}$ & $\epsilon_{FJ}$ & $\textsc{VAR}_{FJ}$ \\ \hline \hline
		$3$ & $100$ & $100$ & $10$     & $0.79011542$    & $0.023361286$       & $0.83428722$    & $0.00522323$        \\ \hline
		$3$ & $100$ & $100$ & $100$    & $0.38049732$    & $0.003518397$       & $0.40706656$    & $0.00123490$        \\ \hline
		$3$ & $100$ & $100$ & $1\,000$ & $0.10418507$    & $0.000494349$       & $0.10331303$    & $0.00011619$        \\ \hline \hline
		$4$ & $100$ & $100$ & $10$     & $0.89923793$    & $0.013469196$       & $0.90575658$    & $0.00365555$        \\ \hline
		$4$ & $100$ & $100$ & $100$    & $0.64715221$    & $0.004050390$       & $0.65129934$    & $0.00117385$        \\ \hline
		$4$ & $100$ & $100$ & $1\,000$ & $0.23606907$    & $0.000409090$       & $0.24419065$    & $0.00008983$        \\ \hline \hline
		$5$ & $100$ & $100$ & $10$     & $0.91908669$    & $0.009465890$       & $0.95246880$    & $0.00215748$        \\ \hline
		$5$ & $100$ & $100$ & $100$    & $0.82803890$    & $0.001517523$       & $0.83137675$    & $0.00062314$        \\ \hline
		$5$ & $100$ & $100$ & $1\,000$ & $0.44637460$    & $0.000352671$       & $0.46965620$    & $0.00004772$        \\ \hline
	\end{tabular}
	\caption{Relative error and variance of the \base approach}	
\end{table}

\clearpage

From the three previously tables we can clearly see that \fcount provide the best approximation for both the indices, 
with high precision and extremely low variance even for $R=10$. 
The \fsamp approach has a lower relative error compared to \base, however the variance between the two approaches are nearly the same.\medskip

We further investigate the variance between \fsamp and \base, this time using $\textsc{IMDb}$ as dataset,
in order to study the stability in a real application.

\begin{table}[h]
	\centering
	\begin{tabular}{|c|c|c|c|c|c|c|c|}
		\cline{5-8}
		\multicolumn{4}{c|}{} & \multicolumn{2}{c|}{\fsamp} & \multicolumn{2}{c|}{\base}\\
		\hline
		$q$ & $|A|$ & $|B|$ & $R$      & $\textsc{VAR}_{BC}$ & $\textsc{VAR}_{FJ}$ & $\textsc{VAR}_{BC}$ & $\textsc{VAR}_{FJ}$ \\ \hline 
		$3$ & $100$ & $100$ & $1\,000$ & $0.00000971$        & $0.00001004$        & $0.00011746$        & $0.00019368$        \\ \hline
		$4$ & $100$ & $100$ & $1\,000$ & $0.00000114$        & $0.00000736$        & $0.00012097$        & $0.00002175$        \\ \hline
		$5$ & $100$ & $100$ & $1\,000$ & $0.00000594$        & $0.00000085$        & $0.00004424$        & $0.00000624$        \\ \hline
		$6$ & $100$ & $100$ & $1\,000$ & $0.00000109$        & $0.00000020$        & $0.00001050$        & $0.00000154$        \\ \hline
	\end{tabular}
	\caption{Variance of \fsamp and \base}	
\end{table}

Now we can see that, for both indices, the variance of \fsamp is one, or in some case two, orders of magnitude fewer compared to \base.

A last test, always comparing \fsamp and \base on $\textsc{IMDb}$, we show some real values of the two indices comparing the ego-networks of the famous comic duo Laurel and Hardy
($q=3$, $R=1\,000$, $|A| = 186$ and $|B| = 415$).

\begin{table}[h]
	\centering
	\begin{tabular}{l|l|l|l|l|}
		\cline{2-5}
		&\multicolumn{2}{c|}{\fsamp} & \multicolumn{2}{c|}{\base}\\
		\cline{2-5}
		& \multicolumn{1}{c|}{BC} & \multicolumn{1}{c|}{FJ} & \multicolumn{1}{c|}{BC} & \multicolumn{1}{c|}{FJ} \\
		\cline{2-5}
		& 0.928940                & 0.780303                & 0.821951                & 0.638258                \\
		& 0.934292                & 0.759470                & 0.730479                & 0.549242                \\
		& 0.929231                & 0.770833                & 0.764780                & 0.575758                \\
		& 0.945752                & 0.787879                & 0.829152                & 0.657197                \\
		& 0.933196                & 0.780303                & 0.758974                & 0.560606                \\
		& 0.950655                & 0.793561                & 0.800000                & 0.621212                \\
		& 0.941658                & 0.761364                & 0.759051                & 0.575758                \\
		& 0.934292                & 0.776515                & 0.801980                & 0.613636                \\
		& 0.933333                & 0.761364                & 0.799020                & 0.617424                \\
		& 0.931282                & 0.768939                & 0.766917                & 0.579545                \\
		\hline
		\multicolumn{1}{|c|}{Mean}     & 0.936167                & 0.774053                & 0.783230                & 0.598864                \\
		\multicolumn{1}{|c|}{Variance} & 0.000055                & 0.000136                & 0.001005                & 0.001265                \\
		\hline
	\end{tabular}
	\caption{Values of estimated BC and FJ setting $A$ and $B$ respectively the movie ego networks of Stan Laurel \& Oliver Hardy }
	\label{table:stanlio}
\end{table}
\subsection*{Fixed relative error}

In this experiment we set the relative error and compare, for each approach, how many paths $R$ we need to reach such relative error.\medskip

\begin{table}[h]
	\centering
	\begin{tabular}{|c|c|c|c|c|c|c|c|c|c|c|}
		\cline{3-11}
		\multicolumn{2}{c|}{} & \multicolumn{3}{c|}{\fcount} & \multicolumn{3}{c|}{\fsamp} & \multicolumn{3}{c|}{\base}\\
		\hline	
		$q$ & $\epsilon$ & R    & T    & VAR      & R         & T    & VAR      & R         & T   & VAR      \\ \hline
		$3$ & $0.20$     & $2$  & $1$  & $0.0725$ & $400$     & $1$  & $0.1194$ & $420$     & $1$ & $0.1150$ \\ \hline
		$3$ & $0.10$     & $3$  & $1$  & $0.0692$ & $1\,000$  & $1$  & $0.0601$ & $900$     & $1$ & $0.1338$ \\ \hline
		$3$ & $0.05$     & $4$  & $1$  & $0.0535$ & $3\,200$  & $1$  & $0.0273$ & $1\,500$  & $1$ & $0.1025$ \\ \hline
		\hline
		$4$ & $0.20$     & $3$  & $2$  & $0.0677$ & $1\,300$  & $1$  & $0.1194$ & $1\,300$  & $1$ & $0.2424$ \\ \hline
		$4$ & $0.10$     & $5$  & $4$  & $0.0532$ & $3\,200$  & $2$  & $0.0992$ & $2\,500$  & $2$ & $0.1806$ \\ \hline
		$4$ & $0.05$     & $10$ & $8$  & $0.0518$ & $8\,000$  & $4$  & $0.0612$ & $7\,900$  & $3$ & $0.1081$ \\ \hline
		\hline
		$5$ & $0.20$     & $5$  & $6$  & $0.0511$ & $5\,000$  & $4$  & $0.1678$ & $6\,000$  & $3$ & $0.2234$ \\ \hline
		$5$ & $0.10$     & $10$ & $18$ & $0.0370$ & $20\,000$ & $12$ & $0.0745$ & $30\,000$ & $8$ & $0.1234$ \\ \hline
		$5$ & $0.05$     & $20$ & $58$ & $0.0204$ & $80\,000$ & $30$ & $0.0376$ & $/$       & $/$ & $/$      \\ \hline
	\end{tabular}
	\caption{Dataset $\textsc{NetInf}$, $|A| = |B| = 100$}
\end{table}
\medskip

We can clearly see that \fcount performed very well, as it needs to sample a very little amount of $q$-paths to reach $\epsilon$.
Instead \fsamp and \base needs many more $q$-paths to reach the precision of \fcount, in particular note that in the last test \base cannot reach the preestablished relative error.\bigskip

\subsection*{Actors' ego-networks}

In order to show a real application easy to understand, we compare
some pairs of actors ego-networks (using \fcount algorithm with $q=3$ and $R=1\,000$):\medskip

\begin{table}[h]
	\centering
	\begin{tabular}{c|c|l|l}
		Actor/actress  & Actor/actress   & BC index & FJ index \\ 
		\hline
		Stan Laurel    & Oliver Hardy    & 0.936167 & 0.774053 \\
		Robert De Niro & Al Pacino       & 0.730935 & 0.231474 \\
		Woody Allen    & Meryl Streep    & 0.556071 & 0.222857 \\
		Meryl Streep   & Roberto Benigni & 0.482909 & 0.160181 \\
		%\hline
	\end{tabular}
\end{table}

\medskip

The values, respect the theory very faithfully for many reasons. \medskip

The Bray-Curtis index, as we already said, is always greater than the Frequency Jaccard and takes more into account the intersection:

the ego-networks of the famous comic duo Laurel and Hardy have a big intersection, this make the Bray-Curtis value very close to 1, however the Frequency-Jaccard is much smaller as Oliver Hardy starred in about $300$ movie without Stan Laurel.

One last observation about the couple Meryl Streep and Roberto Benigni: we have a big difference between the Bray-Curtis and the Frequency Jaccard, this can be due from the fact they are both famous actors (both won the Oscar Prize) who starred with a lot of other famous actors but they haven't starred together.

\subsection*{Parallelization efficiency}

As a last test, we show the parallelization efficiency in the computational time of the color coding table for different numbers of cores used.

\begin{table}[h]
	\centering
	\begin{tabular}{|c|c|c|c|c|c|}
		\cline{3-6}
		\multicolumn{2}{c}{} & \multicolumn{4}{|c|}{ Core } \\ 
		\hline
		Dataset           & $q$  & $24$       & $12$       & $6$        & $3$        \\  \hline \hline
		$\textsc{NetInf}$ & $11$ & $185$      & $199$      & $220$      & $358$      \\  \hline
		$\textsc{NetInf}$ & $13$ & $340$      & $503$      & $948$      & $1753$     \\  \hline
		$\textsc{NetInf}$ & $15$ & $1\,433$   & $2\,235$   & $4\,296$   & $7\,654$   \\  \hline \hline
		$\textsc{IMDb}$   & $3$  & $48\,220$  & $83\,271$  & $139\,107$ & $224\,815$ \\  \hline
		$\textsc{IMDb}$   & $4$  & $105\,943$ & $190\,719$ & $353\,856$ & $684\,342$ \\  \hline
		
	\end{tabular}
\end{table}

In $\textsc{NetInf}$, as it is a small dataset, we see only a slight improvement, 
however in \textsc{IMDb} times (approximately) doubles as the number of cores used is halved.
We can see the times on \textsc{IMDb} in the following plot (logarithmic scale for time, linear scale for cores).\bigskip

\begin{minipage}[h]{0.9\textwidth}
 \centering
 \includegraphics[width=0.9\textwidth]{figure/figure-4-1}
\end{minipage}

\clearpage

\chapter{Conclusion and future works}
    
We presented randomized algorithms and data structures for sketching subgraph similarity, which take into account both the internal structure of subgraphs and their interface to the rest of the network. 
The sketches are relatively small in size (near logarithmic)
and exploit the distributions of the q-grams involved. The proposed
algorithms, f-samp and f-count, guarantee a good approximation
(as unbiased estimators) of the Bray-Curtis index and the Frequency
Jaccard index, and show good practical performance compared to a
less refined baseline sampler. In particular the steady running time
of f-samp on networks with hundreds of millions of edges suggests
its usefulness as an estimator on very large networks.
The assumption that the graph is undirected with one label per
node can be removed, and it would be interesting to study further
similarity indexes that can be sketched with our algorithms

TODO 
\clearpage

\appendix

%!TEX TS-program = pdflatex
%!TEX root = tesi.tex
%!TEX encoding = UTF-8 Unicode

%%%%%%%%%%%%%%%%%%%%%%

\chapter{Code snippets}
  
The code written for this thesis can be found in the personal GitHub page\footnote{\url{https://github.com/GaspareG/SubgraphSimilarity}}.\medskip

\subsection*{Definitions}

Global definitions of some utilities.

\begin{lstlisting}
typedef long long ll;

typedef COLORSET uint32_t  // COLORSET is a bitset of 32 bit
unsigned int N, Q;         // Numer of nodes and length of paths
int color[N];              // Random coloring of nodes
char label[N];             // Labels of nodes
vector<int> G[N];          // Adjacency list for every node in G
map<COLORSET, ll> M[Q][N]; // Color coding table

// Get p-th bit of colorset n
bool getBit(COLORSET n, int p){ return ((n >> p) & 1) == 1; }

// Set p-th bit of colorset n to 1
COLORSET setBit(COLORSET n, int p){ return n |= 1 << p; }

// Reset p-th bit of colorset n to 0
COLORSET clearBit(COLORSET n, int p){ return n &= ~(1 << p); }

// Complementary colorset of n
COLORSET getCompl(COLORSET n){ return ((1 << q) - 1) & (~n); }
\end{lstlisting}

\clearpage
\subsection*{Similarity Indices}

Algorithms \ref{alg:bray-curtis} and \ref{alg:jaccard}

\begin{lstlisting}
double BCW(set<string> W, map<string, ll> freqA, 
                          map<string, ll> freqB) {
	ll num = 0ll;
	ll den = 0ll;
	for (string x : W) {
		ll fax = freqA[x];
		ll fbx = freqB[x];
		num += 2 * min(fax, fbx);
		den += fax + fbx;
	}
	return (double)num / (double)den;
}
\end{lstlisting}

\begin{lstlisting}
double FJW(set<string> W, map<string, ll> freqA,
                          map<string, ll> freqB, ll R) {
	ll num = 0ll;
	for (string x : W) {
		ll fax = freqA[x];
		ll fbx = freqB[x];
		num += min(fax, fbx);
	}
	return (double)num / (double) R;
}
\end{lstlisting}

\clearpage
\subsection*{Color coding}

Implementation of Algorithm \ref{alg:color-coding}

\begin{lstlisting}
void preprocess() {
	#pragma omp parallel for schedule(guided)
	for (unsigned int u = 0; u < N; u++)
		M[1][u][setBit(0, color[u])] = 1ll;
	
	for (unsigned int i = 2; i <= q; i++) {
		#pragma omp parallel for schedule(guided)
		for (unsigned int u = 0; u < N; u++) {
			for (int v : G[u]) {
				for (auto d : M[i - 1][v]) {
					COLORSET s = d.first;
					ll f = d.second;
					if (getBit(s, color[u])) continue;
					ll fp = M[i][u][setBit(s, color[u])];
					M[i][u][setBit(s, color[u])] = f + fp;
				}
			}
		}
	}
}
\end{lstlisting}

\clearpage
\subsection*{Colorful sampling}

Implementation of Algorithms \ref{alg:colorful-sampler} and \ref{alg:random-path-to}

\begin{lstlisting}
set<string> colorfulSampler(vector<int> X, int r) {
	set<string> W;
	set<vector<int>> R;
	vector<ll> freqX;
	for (int x : X) 
		freqX.push_back(M[q][x][getCompl(0)]);
	discrete_distribution<int> distr(freqX.begin(), freqX.end());
	while (R.size() < (size_t)r) {
		int u = X[distr(eng)];
		vector<int> P = randomPathTo(u);
		if (R.find(P) == R.end()) R.insert(P);
	}
	for (auto r : R) {
		reverse(r.begin(), r.end());
		W.insert(L(r));
	}
	return W;
}
\end{lstlisting}

\begin{lstlisting}
vector<int> randomPathTo(int u) {
	vector<int> P;
	P.push_back(u);
	COLORSET D = getCompl(setBit(0l, color[u]));
	for (int i = q - 1; i > 0; i--) {
		vector<ll> freq;
		for (int v : G[u]) 
			freq.push_back(M[i][v][D]);
		discrete_distribution<int> distr(freq.begin(), freq.end());
		#pragma omp critical
		{
			u = G[u][distr(eng)];
		}
		P.push_back(u);
		D = clearBit(D, color[u]);
	}
	reverse(P.begin(), P.end());
	return ret;
}
\end{lstlisting}
\clearpage
\subsection*{Frequency count}

Implementation of Algorithm \ref{alg:f-count}

\begin{lstlisting}
map<string, ll> FCount(set<string> W, multiset<int> X) {
	set<string> WR;
	for (string w : W) {
		reverse(w.begin(), w.end());
		WR.insert(w);
	}
	
	vector<tuple<int, string, COLORSET>> old;
	for (int x : X)
		if (isPrefix(WR, string(&label[x], 1)))
			old.push_back(make_tuple(x, string(&label[x], 1),
                                  setBit(0, color[x])));
	
	for (int i = q - 1; i > 0; i--) {
		vector<tuple<int, string, COLORSET>> current;
		current.clear();
		#pragma omp parallel for schedule(guided)
		for (int j = 0; j < (int)old.size(); j++) {
			auto o = old[j];
			int u = get<0>(o);
			string LP = get<1>(o);
			COLORSET CP = get<2>(o);
			for (int v : G[u]) {
				if (getBit(CP, color[v])) continue;
				COLORSET CPv = setBit(CP, color[v]);
				string LPv = LP + label[v];
				if (!isPrefix(WR, LPv)) continue;
				#pragma omp critical
				{ 
					current.push_back(make_tuple(v, LPv, CPv)); 
				}
			}
		}
		old = current;
	}
	
	map<string, ll> frequency;
	for (auto c : old) {
		string s = get<1>(c);
		reverse(s.begin(), s.end());
		frequency[s]++;
	}
	return frequency;
}
\end{lstlisting}

\clearpage
\subsection*{Frequency sampling}

Implementation of Algorithm \ref{alg:f-samp}

\begin{lstlisting}
map<pair<int, string>, ll> FSamp(vector<int> X, int r) {
	map<pair<int, string>, ll> W;
	set<vector<int>> R;
	vector<ll> freqX;
	freqX.clear();
	for (int x : X) 
		freqX.push_back(M[q][x][getCompl(0)]);
	discrete_distribution<int> distr(freqX.begin(), freqX.end());
	while( R.size() < (size_t)r) {
		int rem = r - R.size();
		#pragma omp parallel for schedule(guided)
		for(int i=0; i<rem; i++) {
			int u;
			#pragma omp critical
			{
				u = X[distr(eng)];
			}
			vector<int> P = randomPathTo(u);
			#pragma omp critical
			{
				R.insert(P);
			}
		}
	}
	for (auto r : R) {
		reverse(r.begin(), r.end());
		W[make_pair(*r.begin(), L(r))]++;
	}
	return W;
}
\end{lstlisting}

\clearpage
\subsection*{F-Count Bray-Curtis}

Implementation of Algorithm \ref{alg:f-count-bc}
\begin{lstlisting}
double FCountBrayCurtis(set<int> A, set<int> B, int tau) {
	multiset<int> mA(A.begin(), A.end());
	multiset<int> mB(B.begin(), B.end());
	multiset<int> X(A.begin(), A.end());
	X.insert(B.begin(), B.end());
	set<string> W = colorfulSampler(X, tau);
	map<string, ll> freqA = FCount(W, mA);
	map<string, ll> freqB = FCount(W, mB);
	return BCW(W, freqA, freqB);
}
\end{lstlisting}

\subsection*{F-Count Frequency Jaccard}

Implementation of Algorithm \ref{alg:f-count-fj}
\begin{lstlisting}
double FCountFrequencyJaccard(set<int> A, set<int> B, int tau) {
	map<string, ll> freqA, freqB;
	set<int> AB(A.begin(), A.end());
	AB.insert(B.begin(), B.end());
	multiset<int> X(AB.begin(), AB.end());
	set<string> W = colorfulSampler(X, tau);
	ll R = 0;
	for(int x : X)
	{
		multiset<int> ma(&x,&x+1);
		map<string, ll> freqAB = processFrequency(W, ma);
		bool inA = A.find(x) != A.end();
		bool inB = B.find(x) != B.end();
		for (auto w : freqAB) {
			string s = w.first;
			ll f = w.second;
			R += f;
			if (inA) freqA[s] += f;
			if (inB) freqB[s] += f;
		}
	}
	return FJW(W, freqA, freqB, R);
}
\end{lstlisting}

\clearpage
\subsection*{F-Sample Bray-Curtis}

Implementation of Algorithm \ref{alg:f-samp-bc}
\begin{lstlisting}
double FSampleBrayCurtis(set<int> A, set<int> B, int tau) {
	multiset<int> X(A.begin(), A.end());
	X.insert(B.begin(), B.end());
	map<pair<int, string>, ll> freqX = FSamp(X, tau);
	map<string, ll> freqA, freqB;
	set<string> W;
	for (auto w : freqX) {
		int u = w.first.first;
		int s = w.first.second;
		ll f = w.second;
		W.insert(s);
		if (A.find(u) != A.end()) freqA[s] += f;
		if (B.find(u) != B.end()) freqB[s] += f;
	}
	return BCW(W, freqA, freqB);
}
\end{lstlisting}

\subsection*{F-Sample Frequency Jaccard}

Implementation of Algorithm \ref{alg:f-samp-fj}
\begin{lstlisting}
double FSampleFrequencyJaccard(set<int> A, 
                               set<int> B, int tau) {
	set<int> AB(A.begin(), A.end());
	AB.insert(B.begin(), B.end());
	vector<int> X(AB.begin(), AB.end());
	map<pair<int, string>, ll>  freqX = FSample(X, tau);
	map<string, ll> freqA, freqB;
	set<string> W;
	ll R = 0;
	for (auto w : freqX) {
		int u = w.first.first;
		int s = w.first.second;
		int f = w.second;
		W.insert(s);
		R += f;
		if (A.find(u) != A.end()) freqA[s] += f;
		if (B.find(u) != B.end()) freqB[s] += f;
	}
	return FJW(W, freqA, freqB, R);
}
\end{lstlisting}


\noindent

\clearpage

\clearpage

\bibliographystyle{plain}
\bibliography{mybib}

% \printindex

%!TEX TS-program = pdflatex
%!TEX root = tesi.tex
%!TEX encoding = UTF-8 Unicode

% \addcontentsline{toc}{chapter}{Ringrazimenti}  

\chapter*{Ringraziamenti}


Desidero ringraziare chi, direttamente o indirettamente, ha contribuito nella realizzazione di questa tesi.\medskip

Un ringraziamento speciale ai miei relatori Roberto Grossi e Andrea Marino con il sostegno di Alessio Conti, per avermi pazientemente seguito, aiutato e consigliato.\medskip

A tutti i miei professori, fonti di grande ispirazione, che hanno contribuito, sotto ogni aspetto, alla mia crescita e formazione.\medskip

Ad Elena, per essermi sempre stata vicina e per avermi sempre supportato, sopportato e consigliato quando ne avevo più bisogno.\medskip

Ai miei genitori che, anche nelle piccole cose, mi sostengono ogni giorno e per tutti i sacrifici che hanno fatto per permettermi di arrivare ad oggi.\medskip

\noindent

\clearpage

\end{document}
