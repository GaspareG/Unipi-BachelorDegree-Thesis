\chapter{Computation of subgraph similarity}

	In this chapter we present four different theoretical algorithm to compute subgraphs similarity as previously defined: an exhaustive enumeration, two similar randomized approach using the tools described in the previous chapter and a naive randomized approach.\\
	
	In the following algorithms, we will make use of parallel instruction, but we leave the specific programming choices and the comparison among the different approaches in the next chapter.
	
% \section{Calculation of indices}
\section{Indices calculation}

Now we illustrate the procedures to calculate the Jaccard and Bray-Curtis indices, as they are independent from the next algorithms we will present.

As previously seen, instead of iterate over all the strings in $\Sigma^{q}$ we can restrict to $\mathcal{L} \subseteq \Sigma^{q}$, the set of all possible $q$-grams found in the $q$-paths of $G$. 

An additional improvement can be made: if we want to calculate the similarity between two set $A, B \subset V$ ranging over $\mathcal{W} = L(A) \cup L(B) \subseteq \Sigma^{q}$ it is enough, as we can easily see that for $x \in ( \Sigma^{q} \setminus \mathcal{W} )$ both $f_A[x]$ and $f_B[x]$ are equal to zero.

\begin{algorithm}
	contenuto...
\end{algorithm}

\begin{algorithm}
	contenuto...
\end{algorithm}

Defined this procedures, in the next algorithms we focus only to compute $\mathcal{W}$ 

\section{Naive approach}

    \begin{algorithm}[h]
    
    \small
    \DontPrintSemicolon
    \SetKwInOut{Input}{Input}
    \SetKwInOut{Output}{Output}
    \Input{\ $G = (V,E)$ undirected graph with $q$ random colors.}
    \Output{\ $(f_{A}[x],f_{B}[x]) $ dynamic programming table for color coding.}
    
    \BlankLine
    
    
    %\textbf{parallel} \ForEach{$u \in V$}{
    %    \ForEach{$v \in N(u)$}{
    %        \ForEach{$\langle C, f \rangle \in M_{i-1,v}$ such that $\chi(u) \not \in C$}{
    %            $f' \gets M_{i,u}\left(C \cup \{\chi(u)\}\right)$\;
    %            $M_{i,u} \gets \langle C \cup \{\chi(u)\}, f' + f \rangle$\;
    %        }
    %    }
    %}   

    \Return{$M$}
    %\setcounter{AlgoLine}{0}
    
    \caption{$preprocess$: \textsc{brute-force}}
    
    \label{alg:color-coding}
    \end{algorithm}

\clearpage

\section{Efficient computation}

\clearpage

\subsection*{Color coding}

    \begin{algorithm}[h]
    
    \small
    \DontPrintSemicolon
    \SetKwInOut{Input}{Input}
    \SetKwInOut{Output}{Output}
    \Input{\ $G = (V,E)$ undirected graph with $q$ random colors.}
    \Output{\ $M = $ dynamic programming table for color coding.}
    
    \BlankLine
    
    \textbf{parallel} \lForEach{$u \in V$}{$M_{1,u} = \langle \chi(u), 1 \rangle$}
    
    \For{$i \in \{ 2, 3, \ldots, q\}$}{
         \textbf{parallel} \ForEach{$u \in V$}{
            \ForEach{$v \in N(u)$}{
                \ForEach{$\langle C, f \rangle \in M_{i-1,v}$ such that $\chi(u) \not \in C$}{
                    $f' \gets M_{i,u}\left(C \cup \{\chi(u)\}\right)$\;
                    $M_{i,u} \gets \langle C \cup \{\chi(u)\}, f' + f \rangle$\;
                }
            }
        }   
    }
    \Return{$M$}
    %\setcounter{AlgoLine}{0}
    
    \caption{$preprocess$: \textsc{color-coding}}
    
    \label{alg:color-coding}
    \end{algorithm}

\clearpage

\subsection*{Colorful sampling}

\clearpage

\subsection*{Frequency count}

\clearpage

\subsection*{Frequency sampling}

\clearpage

\subsection*{Estimating similarity indices}

\clearpage

\section{Baseline algorithm}

\clearpage
